\section{Introduction}

\subsection{Blockchain is more than ledger}
It's known that blockchain is a "distributed ledger".
Bitcoin is the most famous blockchain application.
The distributed ledger is the gimmick of blockchain 1.0.
It led to the first wave of hype around cryptocurrencies.

However, blockchain technology is not only just a ledger or digital currency.
The blockchain technology represented by ethereum inteoduced smart contracts, bringing the 2.0 era of blockchain.
Ethereum is a distributed computer network \autocite{ethereum2014ethereum}.
Each of these nodes will execute transparent code (so-called "smart contracts"), and then store the results on the blockchain,
The execution of the code requires paying a certain percentage of "gas" to the miners who provide computing resources.
This gave birth to DeFi \autocite{chen2020blockchain}, or Decentralized Finance.

Of course, the blockchain also has a wider application in the 3.0 era.
With the deepening of people's understanding of the blockchain, a cyber utopia such as DAO \autocite{wang2019decentralized} has emerged.
The full name of DAO is decentralized autonomous organization.
Different from the traditional corporate form,
DAO is an organization embodied in open and transparent computer code,
controlled only by shareholders and not influenced by the central government.

\subsection{Blockchain can be used in insurance}
Insurance service is also a financial product.
For now, projects such as Nexus and etherisc have designed and implemented blockchain-based insurance, as is shown in figure \ref{fig:etherisc}.
Through the Ethereum platform, people can obtain insurance contracts with smart contracts as the insurance contract \autocite{wan2018pride}.
All the required information for underwriting is contained in the person's private keys.
According to these information, the smart contract will offer the applicants a fair price.
The contract is explicit as its code is open-sourced and can't be modified by anyone.
Once the insured event took place, claims are decided by incentive compatible voting of DAO.

In this paper, we will discuss how blockchain could directly and indirectly improve an insurer’s basic processes and business models.
Therefore, blockchain could improve the customer experience, enhance product value, and lay the groundwork for greater consumer choice in the market.
The end game is to decrease costs, improve operational effectiveness, and strengthen relationships with the insured.
