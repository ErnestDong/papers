\section{Consumer side}

\subsection{Simplifying the application process by making it more client-centric}

According to LIMRA report 2021 (figure \ref{fig:limra}),
those who try to buy individually underwritten life insurance policies on their own are far less pleased with the experience.
Individual applicants are, therefore, more likely to pass on a policy;
partly because they perceive “no current benefit” for enduring such a complicated and often uncomfortable purchasing process.
These kinds of customer experience challenges have been stunting life insurance sales growth.

By joining in the side-chain of the insurer, applicants shares their unchangeable
medical and wellness records with insurer. The smart contract, which is the
pricing model, would offer applicants with a transparent price instantly,
comparing with traditional 45-day application and underwriting timeline.
The applicants could pay ETH or other token to insurer as premium, and might receive
crypto as claims as well. The only charge in such process is the gas in executing
the smart contract and payment, which is usually less than 1\%.
Once applicants decide to leave the network,
the insurer would no longer have access to their private information.

\subsection{Facilitating a dynamic insurer/client relationship}

Establishing strong insurer/client relationships has been problematic.
There may be no interactions except for lapsing or claiming after endless sales call in the life insurance product lifecycle.
In fact, to those who are younger and healthier,
life insurance (and perhaps even health insurance) may appear to be low-return,
high-cost products that offer little relevance in a policyholder’s daily life.

Blockchain could, however, be used not only as a passive secure repository for past medical history,
but also store near-real-time data about the policyholder’s lifestyle and fitness via
IOT devices monitoring their everyday activities. These data could be electronic heath records,
medication, nutrition, lifestyle choices, exercise and even GPS locations.
In this way, insurer can actively maintain insurer/client relationship via health suggestions and cost incentives.
Life insurers could continuously reassess a person’s risk profile and
adjust the cost of coverage accordingly with incentives such as premium adjustments,
discounts for exercise or diet achievements, or even gamification-driven competition.
Health insurers, in turn, could use such telematics data to support wellness programs and
trigger discounts on premiums. In the future era of metaverse, where blockchain would be
the fundamental infrastructure, much more daily activities can be recorded in blockchain.
Therefore, using blockchain can attract more health youth to obtain sufficient insurance service,
and much more health suggestions can be accepted.

Looking at the bigger picture, blockchain and smart contracts
helps build ongoing, interactive, value-added client relationships.
Such a relationship could encourage more individuals to purchase coverage and
stick with their insurer once they do.

\section{Conclusion}

Blockchain could be the catalyst of a profound restructuring in how health and life insurers
access and leverage medical data and other information. It can reduce friction in both insurer side and consumer side.
Ultimately, integrating blockchain applications could give a boost to an insurer’s bottom line by saving money, increasing sales,
and improving retention.
