\section{未来展望:短期增长承压,长期关注竞争}
\textbf{公司基本面与增长逻辑没有发生重大改变}。无论是时代的$\beta$还是公司本身的$\alpha$均未产生重大改变,疫情在短期内推动了Atlassian的成长,在疫情冲击过后公司很难放弃疫情期间产生的数据迁移回过去的工作流,Jira、Confluence等产品粘性依旧很强,云原生时代Atlassian作为先行者仍保持着较强的竞争优势。

\textbf{当前面对的困境主要是短期增长压力}。无论是宏观逆风抑或是企业的短期战略,都会造成短期的压力,短期内营收和用户增长可能会承压。但是行业受宏观影响短时间承压但长期向好,企业预期修复后对云服务支付意愿增强。如果宏观逆风没有超出公司预期,那么公司抢占市场份额的策略可以获得长期的回报。

\textbf{长期关注竞争格局演变与新产品渗透情况}。竞争对手的产品也在不断迭代,开源产品质量也在逐渐提升。来自规模更大、利润更高的公司的激烈竞争可能会对 Atlassian的市场地位造成威胁。由于Atlassian有较高R\&D支出,新产品研发渗透不及预期可能会带来较大的沉没成本带来损失。
\section{风险提示}
\begin{enumerate}
    \item 宏观衰退抑制企业IT支出
    \item 免费用户转化率不足
    \item 企业上云进度不及预期
    \item 竞争不及对手
    \item 新产品研发渗透不及预期
    \item 网络安全风险
\end{enumerate}
