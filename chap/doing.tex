\section*{促进“一带一路”绿色金融发展措施}
\subsection*{统一绿色标准,完善绿色金融基础设施}

“一带一路”国家多为发展中国家,金融体系本就不够发达,绿色金融基础设施建设更是相对滞后。各国在绿色标准、绿色评级、信息披露、统计监测和风险处置等方面存在空白或较大差异,这也是“一带一路”国家绿色金融发展相对较慢的原因。

以绿色金融标准为例,缺乏统一明确的标准定义可能导致绿色资金被用于非绿色项目,从而带来资金使用不当,非绿色项目的“洗绿”及跨国绿色投资风险问题。因此,促进绿色金融的发展,首先需要各国尽快建立与国际接轨的绿色统一标准,完善相关金融基础设施。

中国在助力“一带一路”国家完善绿色金融体系方面大有可为。截至目前,中国已就制订绿色投资原则、绿色金融能力建设等领域与共建“一带一路”国家开展了广泛的合作,例如人民银行在2018年指导发起的发起了《“一带一路”绿色投资原则》,已获得了包括来自巴基斯坦、哈萨克斯坦、蒙古等沿线国家的金融机构认可,中国与“一带一路”国家绿色金融合作空间广阔。

\subsection*{深化国际合作,发挥多边金融机构的支持作用}

多边金融机构对于大部分“一带一路”国家绿色金融发展而言有着重要的支持和引领作用。这其中包括:

\begin{enumerate}
    \item 多边金融机构可直接充当绿色资金的提供方,对“一带一路”国家提供必要的支持和帮助,尤其是投资期限长的大型项目。
    \item 多边金融机构参与对吸引私营部门投资发挥积极作用。多边金融机构投资的项目,在当地政策保险、贷款担保、优惠金融等方面易获得认可,比如,在所罗门群岛的蒂娜河水电开发项目中,绿色气候基金(GCF)提供了1,600万赠款以及40年内7,000万美元贷款的资金,从而使该项目可以满足私营部门投资者的回报期望,吸引私营部门投资进入。
    \item 多边金融机构可提供更多经验支持。大部分“一带一路”国家绿色产业在商业模式、金融工具、减排技术等领域处于探索和创新阶段,多边金融机构可提供更多经验支持,并为创新提供更多可能。
\end{enumerate}

\begin{table}[H]
    \centering
    \begin{tabular}{|c|c|c|}
        \hline
        名称          & 资金规模   & 出资方\\\hline
        气候投资基金(CIF) & 85亿美元  & 14个发达国家      \\
        绿色气候基金(GCF) & 178亿美元 & 24个会员国       \\
        适应基金(AF)    & 8亿美元   & 16个会员国       \\
        全球环境基金(GEF) & 200亿美元 & 184个会员国      \\\hline
    \end{tabular}
    \caption{著名的国际多边气候基金}
\end{table}

\subsection*{以公共部门投入带动私营部门投资者}

2017-2018年全球气候资金来源构成中,私营部门资金占比48\%,公共部门占比52\%,但是具体到“一带一路”国家而言,大部分绿色投资更多来源于政府和多边金融机构,私营部门投资者参与相对较少。推动绿色融资,公共部门也可以通过产品和机制创新调动私营部门参与“一带一路”绿色金融的积极性,包括公共资金担保、政府绿色采购等。

以肯尼亚LakeTurkana风电场项目为例,2014年非洲开发银行联合渣打银行向该项目提供了2千万欧元的担保,极大提振了私营部门投资者的信心,项目得以完成6.25亿欧元的筹资目标并于2018年顺利完工。

在私营部门投资者中,尤以机构投资者最为重要,包括养老基金、保险公司、对冲基金和共同基金。机构投资者的参与能够为绿色金融市场提供流动性,并为原来的债权方如商业银行打通退出渠道,进一步激活绿色投融资市场。然而,机构投资者投资于绿色金融产品的案例仍主要见于发达国家,例如瑞银在欧洲设立的全球最大绿色股权基金,而大部分发展中国家相对较少。未来,“一带一路”国家可考虑通过税收优惠、风险补偿、绩效评价等多方面给予优惠,尽可能吸引机构投资者参与绿色“一带一路”投资当中。

对于“一带一路”国家而言,实现绿色转型单纯依靠债权融资容易加重其还款付息压力,可能带来更高的债务风险。此外,考虑到这些国家的绿色转型刚刚起步,风险与机遇并存,其它诸如绿色股权、保险等金融工具更有利于促进国际间风险分担、提供长期资金。因此,“一带一路”绿色金融可借鉴国际先进经验,完善风险评估机制,丰富金融工具,以吸引更多形式的资金来源。

以中国为例,据美国企业研究所(AEI)的数据,中国对“一带一路”国家投资主要集中在能源和交通运输领域,特别地,2020年,中国对“一带一路”国家可再生能源投资占比达57\%,首次超过传统化石能源。随着“一带一路”国家绿色金融工具的不断发展和完善,预计中国股权类绿色投资规模有望进一步提升,从而更好地促进“一带一路”国家低碳转型发展。

\subsection*{将绿色发展理念嵌入更多金融合作进程}
以债务问题处置为例,2020年新冠疫情爆发后,对于发展中国家原本高企的对外债务压力更是雪上加霜,出现了普遍的流动性困难。2020年2月以来,共有40个国家和地区遭信用评级机构降级,已有6个国家相继违约。为应对这一局面,国际社会纷纷采取措施协助发展中国家渡过难关,包括债务暂缓偿付和G20债务处理共同框架。但国际社会要最终解决这一问题,仍需推出一套系统化的处置方案。

在解决“一带一路”发展中国家债务问题的过程当中,绿色金融同样可占据一席之地。一方面,债权方可在重组方案中采用债务气候转换、债务环境转换等工具,即要求债务国将部分减免额度再投资于气候、环境领域,还可允许债务国基于绿色项目发行绿色证券置换违约债务。通过赋予绿色属性,将债务处置与推动“一带一路”国家绿色转型相结合。另一方面,对于存量“一带一路”项目,可在甄别其绿色特征或绿色潜力基础上,基于项目现金流进行绿色资产证券化,或对国际投资者发行绿色债券,这也有助于推动存量项目的绿色转型。通过上述操作,本轮债务处置方案融入了绿色发展理念,更好与实现长期气候目标相契合。
