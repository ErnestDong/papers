% Created 2022-03-26 Sat 17:11
% Intended LaTeX compiler: xelatex
\documentclass[a4paper,12pt]{ctexart}
\usepackage{xeCJK}
\usepackage[backend = biber, style = gb7714-2015, url=false,gbtitlelink=true]{biblatex}

\usepackage{booktabs}
\usepackage{enumerate}
\usepackage{indentfirst}
\usepackage{graphicx}
\usepackage{longtable}
\usepackage[normalem]{ulem}
\usepackage{amsmath}
\usepackage{amsfonts}
\usepackage{amssymb}
\usepackage{float}
\usepackage{capt-of}
\usepackage[hidelinks]{hyperref}
\usepackage[table,xcdraw]{xcolor}
\author{董晨阳 \thanks{学号1800015446} 许天朗 李昭伦}
\date{\today}
\title{风管模型第一次作业}
\hypersetup{
 pdfauthor={董晨阳 \thanks{学号1800015446} 许天朗 李昭伦},
 pdftitle={风管模型第一次作业},
 pdfkeywords={},
 pdfsubject={},
 pdfcreator={Emacs 29.0.50 (Org mode 9.6)}, 
 pdflang={English}}
\begin{document}

\maketitle
\section{假设}
\label{sec:org5da1f3c}
\begin{itemize}
\item 资产的收益率服从正态分布,资产价格服从对数正态分布。
\item 无风险利率 \(r=0.05\)。
\item 可以通过融券等方式裸卖空,且 haircut 为 0。
\end{itemize}
\section{标的及理由}
\label{sec:orgfdcebd3}
我们的理念是坚持价值投资。选择标的的标准是股票之间相关性较低,不容易因疫情或者俄乌冲突等黑天鹅事件冲击到基本面,有长期持有价值的公司。基于此理念,本文将从以下几个行业中选择行业龙头公司。
\subsection{新能源行业}
\label{sec:org9a2787e}
新能源无疑是近两三年以来最火热的赛道,亦是不少基金获得市场超额收益的来源。尽管新能源行业估值较高,但在燃油车被新能源车替代的大背景下,我们相信新能源渗透率将进一步提升,高估值是市场对于未来增长前景的估计,新能源赛道上获得超额收益的可能性仍然很大。当前新能源车是以电动车为主的,在可预见的几年内,电池占整车成本较高是相对确定的,新能源行业仍有右侧机会,我们建议超配。

本文将着重关注比亚迪。整车方面,其混动车销量遥遥领先于其他厂家,出爆款的历史较多,且全流程技术自主可控,为少数大幅盈利的电动车为主的主机厂;电池方面磷酸铁锂电池技术领先,且即将外供;车用芯片部门即将分拆上市,亦可对母公司财务有一定的帮助。估值相对历史最高已降至 67\% 分位,我们相信公司股价受市场情绪波动而低估。

此外,宁德时代作为三元电池的龙头,我们也计划对其持有一定仓位。宁德时代三元电池市占率达 57.17\% ,我们认为宁德时代将是电车时代的台积电,未来重要性将进一步上升。
\subsection{养殖业}
\label{sec:org87d469f}
历史上猪企的股票表现随猪周期呈现出 48 个月左右波动。受俄乌冲突影响,全球玉米减少,豆粕期货已经上涨了相当幅度,势必会传导到下游的饲料和养殖行业。当前猪周期仍处于二次探底的阶段,22 省猪肉价格已跌至 13 元,养殖户深度亏损出清,新一轮猪周期或提前启动,目前或是养殖业是左侧布局的最佳时期。我们亦建议对养殖业超配。

随着财报季的到来,猪企大幅亏损概率较大,是否能把握住周期是猪企面对的巨大考验。上一轮猪周期中牧原股份对时机把握精准,超越温氏股份成为养猪龙头。因此我们计划布局行业前二的牧原股份和温氏股份,以期其能在景气上行期抢占市场份额。
\section{代码说明}
\label{sec:orgad42289}
开发环境: macOS12.3 + python 3.9

推荐运行方式:
\begin{verbatim}
pip install poetry && poetry shell
poetry shell
python ./tests/test_src.py
\end{verbatim}
结果保存在 result.log 中,内容为

\begin{verbatim}
companys:牧原股份, 招商银行, 中国平安, 东方财富, 中国中免, 比亚迪, 温氏股份, 宁德时代, 中国建筑
proportion:0.05716406332367887,-0.011507889327569928,-0.09348900683417324,0.20134989514587248,0.040865422930575025,0.0861138840584174,0.3197216111374898,0.3572151189497322,0.04256690061597743
VaR src.var_cov in 10 days -1024853.2797570882
VaR src.var_cov in 60 days -2510367.5966226864
companys:牧原股份, 招商银行, 中国平安, 东方财富, 中国中免, 比亚迪, 温氏股份, 宁德时代, 中国建筑
proportion:0.05716406332367887,-0.011507889327569928,-0.09348900683417324,0.20134989514587248,0.040865422930575025,0.0861138840584174,0.3197216111374898,0.3572151189497322,0.04256690061597743
VaRsrc.historical in 10 days -1107811.5022449195
VaRsrc.historical in 60 days -2713572.911686154
companys:牧原股份, 招商银行, 中国平安, 东方财富, 中国中免, 比亚迪, 温氏股份, 宁德时代, 中国建筑
proportion:0.05716406332367887,-0.011507889327569928,-0.09348900683417324,0.20134989514587248,0.040865422930575025,0.0861138840584174,0.3197216111374898,0.3572151189497322,0.04256690061597743
VaR src.monte_carloin 10 days -1602219.678611719
VaR src.monte_carloin 60 days -4194253.1065305998
companys:牧原股份, 招商银行, 中国平安, 东方财富, 中国中免, 比亚迪, 温氏股份, 宁德时代, 中国建筑
proportion:0.05716406332367887,-0.011507889327569928,-0.09348900683417324,0.20134989514587248,0.040865422930575025,0.0861138840584174,0.3197216111374898,0.3572151189497322,0.04256690061597743
\end{verbatim}


代码文档见 \url{./doc/develop.md}
\end{document}
