% Created 2022-03-28 Mon 23:16
% Intended LaTeX compiler: xelatex
\documentclass[a4paper,12pt]{ctexart}
\usepackage{xeCJK}
\usepackage[backend = biber, style = gb7714-2015, url=false,gbtitlelink=true]{biblatex}

\usepackage{booktabs}
\usepackage{enumerate}
\usepackage{indentfirst}
\usepackage{graphicx}
\usepackage{longtable}
\usepackage[normalem]{ulem}
\usepackage{amsmath}
\usepackage{amsfonts}
\usepackage{amssymb}
\usepackage{float}
\usepackage{capt-of}
\usepackage[hidelinks]{hyperref}
\usepackage[table,xcdraw]{xcolor}
\author{董晨阳 \thanks{学号1800015446} 许天朗 \thanks{学号1800015428} 李昭伦 \thanks{学号1800015469}}
\date{\today}
\title{风管模型第一次作业}
\hypersetup{
 pdfauthor={董晨阳 \thanks{学号1800015446} 许天朗 李昭伦 \thanks{1800015469}},
 pdftitle={风管模型第一次作业},
 pdfkeywords={},
 pdfsubject={},
 pdfcreator={Emacs 29.0.50 (Org mode 9.6)}, 
 pdflang={English}}
\begin{document}

\maketitle
\section{假设}
\label{sec:orgef30c16}
\begin{itemize}
\item 资产的收益率服从正态分布,资产价格服从对数正态分布。
\item 无风险年化利率 \(r=0.05\)。
\item 可以通过融券等方式裸卖空,且 haircut 为 0。
\end{itemize}
\section{标的及理由}
\label{sec:org4fadadd}
我们的理念是坚持价值投资。选择标的的标准是股票之间相关性较低,不容易因疫情或者俄乌冲突等黑天鹅事件冲击到基本面,有长期持有价值的公司。基于此理念,本文将从以下几个行业中选择行业龙头公司。
\subsection{新能源行业}
\label{sec:org6091fb3}
新能源无疑是近两三年以来最火热的赛道,亦是不少基金获得市场超额收益的来源。尽管新能源行业估值较高,但在燃油车被新能源车替代的大背景下,我们相信新能源渗透率将进一步提升,高估值是市场对于未来增长前景的估计,新能源赛道上获得超额收益的可能性仍然很大。当前新能源车是以电动车为主的,在可预见的几年内,电池占整车成本较高是相对确定的,新能源行业仍有右侧机会,我们建议超配。

本文将着重关注比亚迪。整车方面,其混动车销量遥遥领先于其他厂家,出爆款的历史较多,且全流程技术自主可控,为少数大幅盈利的电动车为主的主机厂;电池方面磷酸铁锂电池技术领先,且即将外供;车用芯片部门即将分拆上市,亦可对母公司财务有一定的帮助。估值相对历史最高已降至 67\% 分位,我们相信公司股价受市场情绪波动而低估。

此外,宁德时代作为三元电池的龙头,我们也计划对其持有一定仓位。宁德时代三元电池市占率达 57.17\% ,我们认为宁德时代将是电车时代的台积电,未来重要性将进一步上升。
\subsection{养殖业}
\label{sec:orgf891dbf}
历史上猪企的股票表现随猪周期呈现出 48 个月左右波动。受俄乌冲突影响,全球玉米减少,豆粕期货已经上涨了相当幅度,势必会传导到下游的饲料和养殖行业。当前猪周期仍处于二次探底的阶段,22 省猪肉价格已跌至 13 元,养殖户深度亏损出清,新一轮猪周期或提前启动,目前或是养殖业是左侧布局的最佳时期。我们亦建议对养殖业超配。

随着财报季的到来,猪企大幅亏损概率较大,是否能把握住周期是猪企面对的巨大考验。上一轮猪周期中牧原股份对时机把握精准,超越温氏股份成为养猪龙头。因此我们计划布局行业前二的牧原股份和温氏股份,以期其能在景气上行期抢占市场份额。
\subsection{金融业}
\label{sec:org87a0558}
改革开放使金融业走上了蓬勃发展的轨道。目前,我国已基本建立了以国有金融机构为主体、各类金融机构分工合作的金融组织体系,逐步形成了银行、证券、保险业分业经营、分业监管的金融体制。我们从银行、证券、保险业分别选取一家具有长期前景的企业加入投资组合。

在银行业方面,本文着重关注招商银行。招商银行于3月18日发布2021年报,2021年归母净利润、营业收入、拨备前利润同比增速分别为23.2\%、14.0\%、14.1\%。营收、归母净利润同比增速均创近5年以来最佳记录,驱动因素主要包括资产投放提速、息差环比提升、非息收入亮眼、信用成本下行等。2021年年化ROE、ROA分别同比增长1.23\%、0.12\%至16.96\%、1.36\%,呈现较强上升趋势。大财富管理收入(即财富+资管+托管)同比增长33.9\%,在营收中占比近16\%。零售、金葵花及以上客户数同比增速均较9月末提升,财富门槛降低,触达更广客群。我们认为招商银行在业内率先从运营端入手革新大财富业态,有望加速从“卖方模式”向“买方模式”升级、从“销售导向”向“生态经营”升级,构筑长期坚固护城河。

在证券业方面,选取的是东方财富。根据该公司年报,2021年收入130.94亿元,同比增长58.9\%;归母净利85.53亿元,同比增长79.0\%。年报较关键的数据为基金销量数据,据年报,2021年天天基金非货币基金销售规模1.34万亿元,同增92\%,超过预期。2018到2021年东财净利增长近10倍(从9.6亿元增长至86亿元),其原因包括中国财富管理规模的大幅增长及公司业务运营能力的凸显。东财有望受益于产业变革带来的竞争格局重构。

在保险业方面,我们关注中国平安。中国平安在今年展示了其推进寿险业务转型的决心。公司2021年新业务价值(NBV)疲软,但营运利润较好,其管理层强调改革已取得进展。公司寿险营业部中的30\%已完成数字化转型,预计其余营业部的转型工作将在2022年完成。改革是中国平安寿险业务恢复增长动力的关键,中国平安是最早专注于业务模式转型升级的保险公司之一,有望比竞争对手更早完成转型。由于疲弱的NBV增长和严重的房地产投资亏损,中国平安的估值在2021年不断下滑,但我们认为目前估值已反映了大部分负面因素,且市场预期并不高,这为潜在股价上涨奠定了基础。

\subsection{消费及旅游行业}
\label{sec:orgf8c98a9}
根据2020年及2021年的经验,夏季气温升高,新冠疫情的传播会相对减弱,消费行业及旅游业会呈现出复苏态势。从长期来看,近两年收到疫情的影响,消费和旅游行业受到较大冲击,我们认为人类与新冠共存将成为一个必然的趋势,随着疫苗水平及救治手段地不断进步,长期内消费的需求将得到释放。因此我们看好消费及旅游行业将呈现增长的趋势。

中国免税品有限责任公司成立于1984年,是经中国国务院批准在全国范围内开展免税业务的国有公司。中免集团自成立以来,始终秉承“分享购物的快乐、延伸旅游的享受”的企业使命,发展成为中国免税行业的代表和旗舰企业,是中国最大的奢侈品运营商。受到近期疫情严峻态势的影响,目前中国中免股票处于低估值时期,我们相信在疫情得到控制之后,消费及旅游的需求得到释放,中国中免的股票将会上涨。

\subsection{建筑行业}
\label{sec:orgb7ee309}
展望2022年,建筑行业的景气度整体由固定资产投资驱动。拆分来看,建筑行业的子板块包括基础建设、园林工程、房屋建设、装修装饰、专业工程等,不同细分板块的景气度分别由基建投资、地产投资、制造业投资等所驱动。且建筑行业工业化、数字化、智能化的“三化”变革带来的效率提升在逐步深化,存量房的累积、集中度的提升也带来了新的机遇,叠加行业估值位于历史底部,我们认为建筑行业2022年仍有较多的结构性投资机会可供挖掘。

中国建筑集团有限公司组建于1982年,其前身为原国家建工总局,是为数不多的不占有大量的国家投资,不占有国家的自然资源和经营专利,以从事完全竞争性的建筑业和地产业为核心业务而发展壮大起来的国有重要骨干企业。在大基建的背景下,我们认为中国建筑拥有结构性增长的投资机会。
\section{代码说明}
\label{sec:org9154238}
\subsection{环境}
\label{sec:org4f6cb81}

开发环境: macOS12.3 + python 3.9

将数据文件移动到 \url{./lib/} 文件夹下,推荐运行方式:
\begin{verbatim}
pip install poetry && poetry shell
poetry shell
python ./tests/test_src.py
\end{verbatim}
项目的配置见 \url{./src/config.py} ,可在其中设定无风险利率、数据文件路径、初始头寸等数据。代码运行结果通过单元测试的结果展示,结果会保存在 result.log 中,内容为

\begin{verbatim}
variance-covariance VaR in 10 days -1024853.2797570882
variance-covariance VaR in 60 days -2510367.5966226864
company 牧原股份 with proportion -1.15 %
company 招商银行 with proportion 8.61 %
company 中国平安 with proportion 31.97 %
company 东方财富 with proportion -9.35 %
company 中国中免 with proportion 4.26 %
company 比亚迪 with proportion 5.72 %
company 温氏股份 with proportion 20.13 %
company 宁德时代 with proportion 4.09 %
company 中国建筑 with proportion 35.72 %
Historical VaR in 10 days -1107811.5022449195
Historical VaR in 60 days -2713572.911686154
monte-carlo VaR in 10 days -1542048.9237156862
monte-carlo VaR in 60 days -3851415.7985102646
\end{verbatim}


详细的代码文档如其他安装方式、各文件的设计目的及使用方法见 \url{./doc/develop.md}

\subsection{文件读取}
\label{sec:orga8dd2f2}
数据文件来自于 wind 客户端,亦见数据文件中的“数据来源:wind”。读取文件的代码在 \url{./src/config.py} 中,采用前复权的价格进行回测。

\subsection{头寸确定}
\label{sec:org86bbe05}
本文采用 Markowitz 有效前沿理论确定相关头寸,完整代码见 \url{./src/markowitz.py} ,确定的有效前沿如图\ref{fig:orgff223c7}所示:
\small\begin{verbatim}
import matplotlib.pyplot as plt
import numpy as np
import pandas as pd
import seaborn as sns
from scipy import linalg

from src.config import expected_return, initial_investment, prepare_data


class Markowitz:
    def __init__(self, returns):
        self.returns = returns
        self.cov = returns.cov()
        self.company = returns.columns

    def solveMinVar(self, expected_return):
        cov = np.array(self.cov)
        mean = np.array(self.returns.mean())
        row1 = np.append(
            np.append(cov.swapaxes(0, 1), [mean], axis=0),
            [np.ones(len(mean))],
            axis=0
        ).swapaxes(0, 1)
        row2 = list(np.ones(len(mean)))
        row2.extend([0, 0])
        row3 = list(mean)
        row3.extend([0, 0])
        A = np.append(row1, np.array([row2, row3]), axis=0)
        b = np.append(np.zeros(len(mean)), [1, expected_return], axis=0)
        results = linalg.solve(A, b)

        return pd.DataFrame(results[:-2], index=self.company)

    def calVar(self, portion):
        portion = portion.values
        return np.dot(np.dot(portion.T, self.cov), portion)[0]

    def plotFrontier(self):
        expected_return = [x / 100000 for x in range(-500, 1000)]
        variance = list(
            map(
                lambda x: self.calVar(self.solveMinVar(x)),
                expected_return,
            )
        )
        sns.set()
        plt.plot(variance, expected_return)
        plt.xlabel("Variance")
        plt.ylabel("Expected Return")
        plt.title("Efficient Frontier")
        plt.show()


if __name__ == "__main__":
    companies, data = prepare_data()
    markowitz = Markowitz(data)
    markowitz.plotFrontier()

\end{verbatim}
\begin{figure}[htbp]
\centering
\includegraphics[width=.9\linewidth]{./.ob-jupyter/4e0e7bb391686cabff8e4fd95bf5a5352dca2255.png}
\caption{\label{fig:orgff223c7}Markowitz Effient Frontier}
\end{figure}

在无风险利率为 5\% 的前提下,最优的购买比例为
\begin{verbatim}
proportion = markowitz.solveMinVar(expected_return/365)
[companies[i][0] for i in proportion.index], [round(i, 4) for i in proportion[0]]
\end{verbatim}

\begin{center}
\begin{tabular}{rrrrrrrrr}
比亚迪 & 牧原股份 & 东方财富 & 温氏股份 & 宁德时代 \\
0.0516 & -0.0129 & -0.0957 & 0.2051 & 0.0193 \\
招商银行 & 中国平安 & 中国建筑 & 中国中免\\
0.0521 & 0.3665 & 0.3774 & 0.0366\\
\end{tabular}
\end{center}

\subsection{VaR 的计算}
\label{sec:org6da6ec7}
不同方法计算的 VaR 代码分别位于 \url{./src/} 文件下,对外均暴露借口 \texttt{var} 方法,
除蒙特卡洛模拟外均采用一致的参数接口
\texttt{var(data:pd.DataFrame, proportion:dict, days:int=10, quantile:float=99)} 。
\texttt{data} 为收益率数据, \texttt{proportion} 为持仓分布, \texttt{days} 为计算 VaR 的时间区间, \texttt{quantile} 为置信度
\subsubsection{历史模拟法}
\label{sec:orge19e605}
代码实现详见 \url{./src/historical.py} ,利用 \texttt{pandas} api 计算一日亏损的 99\%分位点得到。
\subsubsection{方差-协方差法}
\label{sec:org701877a}
代码实现详见 \url{./src/var\_cov.py} ,首先检验收益率分布是否满足正态分布,如不服从会报 \texttt{AssertionError} 。
\subsubsection{蒙特卡洛模拟法}
\label{sec:org4def993}
蒙特卡洛模拟法需要额外的三个参数 \texttt{n1} 、 \texttt{n2} 和 \texttt{price} ,
分别为模拟当天收益率时的时间分段数、模拟计算当天收益率的生成次数,
以及当前时点的价格。
\section{unittests}
\label{sec:org2908b31}
本文设计了单元测试,降低重复 IO 次数,在努力保证代码结构清晰的同时,
方便老师和助教检验结果,使用方法为
\begin{verbatim}
python tests/test_src.py
\end{verbatim}

若数据文件出错、python 版本出错,抑或是代码运行不符合预期,单元测试会显示不通过。
若通过,则在项目根目录下会出现 \url{./result.log} ,文件内容即为本次作业计算结果。
\end{document}
