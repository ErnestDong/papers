% Created 2022-03-28 Mon 20:32
% Intended LaTeX compiler: xelatex
\documentclass[a4paper,12pt]{ctexart}
\usepackage{xeCJK}
\usepackage[backend = biber, style = gb7714-2015, url=false,gbtitlelink=true]{biblatex}

\usepackage{booktabs}
\usepackage{enumerate}
\usepackage{indentfirst}
\usepackage{graphicx}
\usepackage{longtable}
\usepackage[normalem]{ulem}
\usepackage{amsmath}
\usepackage{amsfonts}
\usepackage{amssymb}
\usepackage{float}
\usepackage{capt-of}
\usepackage[hidelinks]{hyperref}
\usepackage[table,xcdraw]{xcolor}
\author{董晨阳 \thanks{学号1800015446} 许天朗 李昭伦}
\date{\today}
\title{风管模型第一次作业}
\hypersetup{
 pdfauthor={董晨阳 \thanks{学号1800015446} 许天朗 李昭伦},
 pdftitle={风管模型第一次作业},
 pdfkeywords={},
 pdfsubject={},
 pdfcreator={Emacs 29.0.50 (Org mode 9.6)}, 
 pdflang={English}}
\begin{document}

\maketitle
\section{假设}
\label{sec:org9169128}
\begin{itemize}
\item 资产的收益率服从正态分布,资产价格服从对数正态分布。
\item 无风险年化利率 \(r=0.05\)。
\item 可以通过融券等方式裸卖空,且 haircut 为 0。
\end{itemize}
\section{标的及理由}
\label{sec:orgd733dea}
我们的理念是坚持价值投资。选择标的的标准是股票之间相关性较低,不容易因疫情或者俄乌冲突等黑天鹅事件冲击到基本面,有长期持有价值的公司。基于此理念,本文将从以下几个行业中选择行业龙头公司。
\subsection{新能源行业}
\label{sec:org96bce39}
新能源无疑是近两三年以来最火热的赛道,亦是不少基金获得市场超额收益的来源。尽管新能源行业估值较高,但在燃油车被新能源车替代的大背景下,我们相信新能源渗透率将进一步提升,高估值是市场对于未来增长前景的估计,新能源赛道上获得超额收益的可能性仍然很大。当前新能源车是以电动车为主的,在可预见的几年内,电池占整车成本较高是相对确定的,新能源行业仍有右侧机会,我们建议超配。

本文将着重关注比亚迪。整车方面,其混动车销量遥遥领先于其他厂家,出爆款的历史较多,且全流程技术自主可控,为少数大幅盈利的电动车为主的主机厂;电池方面磷酸铁锂电池技术领先,且即将外供;车用芯片部门即将分拆上市,亦可对母公司财务有一定的帮助。估值相对历史最高已降至 67\% 分位,我们相信公司股价受市场情绪波动而低估。

此外,宁德时代作为三元电池的龙头,我们也计划对其持有一定仓位。宁德时代三元电池市占率达 57.17\% ,我们认为宁德时代将是电车时代的台积电,未来重要性将进一步上升。
\subsection{养殖业}
\label{sec:org7387dee}
历史上猪企的股票表现随猪周期呈现出 48 个月左右波动。受俄乌冲突影响,全球玉米减少,豆粕期货已经上涨了相当幅度,势必会传导到下游的饲料和养殖行业。当前猪周期仍处于二次探底的阶段,22 省猪肉价格已跌至 13 元,养殖户深度亏损出清,新一轮猪周期或提前启动,目前或是养殖业是左侧布局的最佳时期。我们亦建议对养殖业超配。

随着财报季的到来,猪企大幅亏损概率较大,是否能把握住周期是猪企面对的巨大考验。上一轮猪周期中牧原股份对时机把握精准,超越温氏股份成为养猪龙头。因此我们计划布局行业前二的牧原股份和温氏股份,以期其能在景气上行期抢占市场份额。
\section{代码说明}
\label{sec:org23c397c}
\subsection{环境}
\label{sec:orgd9fef59}

开发环境: macOS12.3 + python 3.9

将数据文件移动到 \url{./lib/} 文件夹下,推荐运行方式:
\begin{verbatim}
pip install poetry && poetry shell
poetry shell
python ./tests/test_src.py
\end{verbatim}
项目的配置见 \url{./src/config.py} ,可在其中设定无风险利率、数据文件路径、初始头寸等数据。代码运行结果通过单元测试的结果展示,结果会保存在 result.log 中,内容为

\begin{verbatim}
variance-covariance VaR in 10 days -1024853.2797570882
variance-covariance VaR in 60 days -2510367.5966226864
company 牧原股份 with proportion -1.15 %
company 招商银行 with proportion 8.61 %
company 中国平安 with proportion 31.97 %
company 东方财富 with proportion -9.35 %
company 中国中免 with proportion 4.26 %
company 比亚迪 with proportion 5.72 %
company 温氏股份 with proportion 20.13 %
company 宁德时代 with proportion 4.09 %
company 中国建筑 with proportion 35.72 %
Historical VaR in 10 days -1107811.5022449195
Historical VaR in 60 days -2713572.911686154
monte-carlo VaR in 10 days -1542048.9237156862
monte-carlo VaR in 60 days -3851415.7985102646
\end{verbatim}


详细的代码文档如其他安装方式、各文件的设计目的及使用方法见 \url{./doc/develop.md}

\subsection{文件读取}
\label{sec:org5f409df}
数据文件来自于 wind 客户端,亦见数据文件中的“数据来源:wind”。读取文件的代码在 \url{./src/config.py} 中,采用前复权的价格进行回测。

\subsection{头寸确定}
\label{sec:orga4508df}
本文采用 Markowitz 有效前沿理论确定相关头寸,完整代码见 \url{./src/markowitz.py} ,确定的有效前沿如图\ref{fig:org5fe265c}所示:
\begin{verbatim}
import matplotlib.pyplot as plt
import numpy as np
import pandas as pd
import seaborn as sns
from scipy import linalg

from src.config import expected_return, initial_investment, prepare_data


class Markowitz:
    def __init__(self, returns):
        self.returns = returns
        self.cov = returns.cov()
        self.company = returns.columns

    def solveMinVar(self, expected_return):
        cov = np.array(self.cov)
        mean = np.array(self.returns.mean())
        row1 = np.append(
            np.append(cov.swapaxes(0, 1), [mean], axis=0), [np.ones(len(mean))], axis=0
        ).swapaxes(0, 1)
        row2 = list(np.ones(len(mean)))
        row2.extend([0, 0])
        row3 = list(mean)
        row3.extend([0, 0])
        A = np.append(row1, np.array([row2, row3]), axis=0)
        b = np.append(np.zeros(len(mean)), [1, expected_return], axis=0)
        results = linalg.solve(A, b)

        return pd.DataFrame(results[:-2], index=self.company)

    def calVar(self, portion):
        portion = portion.values
        return np.dot(np.dot(portion.T, self.cov), portion)[0]

    def plotFrontier(self):
        expected_return = [x / 100000 for x in range(-500, 1000)]
        variance = list(
            map(
                lambda x: self.calVar(self.solveMinVar(x)),
                expected_return,
            )
        )
        sns.set()
        plt.plot(variance, expected_return)
        plt.xlabel("Variance")
        plt.ylabel("Expected Return")
        plt.title("Efficient Frontier")
        plt.show()


if __name__ == "__main__":
    companies, data = prepare_data()
    markowitz = Markowitz(data)
    markowitz.plotFrontier()

\end{verbatim}
\begin{figure}[htbp]
\centering
\includegraphics[width=.9\linewidth]{./.ob-jupyter/4e0e7bb391686cabff8e4fd95bf5a5352dca2255.png}
\caption{\label{fig:org5fe265c}Markowitz Effient Frontier}
\end{figure}

在无风险利率为 5\% 的前提下,最优的购买比例为
\begin{verbatim}
proportion = markowitz.solveMinVar(expected_return/365)
[companies[i][0] for i in proportion.index], [round(i, 4) for i in proportion[0]]
\end{verbatim}

\begin{center}
\begin{tabular}{rrrrrrrrr}
比亚迪 & 牧原股份 & 东方财富 & 温氏股份 & 宁德时代 & 招商银行 & 中国平安 & 中国建筑 & 中国中免\\
0.0516 & -0.0129 & -0.0957 & 0.2051 & 0.0193 & 0.0521 & 0.3665 & 0.3774 & 0.0366\\
\end{tabular}
\end{center}

\subsection{VaR 的计算}
\label{sec:org10d3c74}
不同方法计算的 VaR 代码分别位于 \url{./src/} 文件下,对外均暴露借口 \texttt{var} 方法,
除蒙特卡洛模拟外均采用一致的参数接口
\texttt{var(data:pd.DataFrame, proportion:dict, days:int=10, quantile:float=99)} 。
\texttt{data} 为收益率数据, \texttt{proportion} 为持仓分布, \texttt{days} 为计算 VaR 的时间区间, \texttt{quantile} 为置信度
\subsubsection{历史模拟法}
\label{sec:org9ea8ef2}
代码实现详见 \url{./src/historical.py} ,利用 \texttt{pandas} api 计算一日亏损的 99\%分位点得到。
\subsubsection{方差-协方差法}
\label{sec:org52abd2b}
代码实现详见 \url{./src/var\_cov.py} ,首先检验收益率分布是否满足正态分布,如不服从会报 \texttt{AssertionError} 。
\subsubsection{蒙特卡洛模拟法}
\label{sec:orgd21aea7}
蒙特卡洛模拟法需要额外的三个参数 \texttt{n1} 、 \texttt{n2} 和 \texttt{price} ,
分别为模拟当天收益率时的时间分段数、模拟计算当天收益率的生成次数,
以及当前时点的价格。
\section{unittests}
\label{sec:org22fa7a8}
本文设计了单元测试,降低重复 IO 次数,在努力保证代码结构清晰的同时,
方便老师和助教检验结果,使用方法为
\begin{verbatim}
python tests/test_src.py
\end{verbatim}

若数据文件出错、python 版本出错,抑或是代码运行不符合预期,单元测试会显示不通过。
若通过,则在项目根目录下会出现 \url{./result.log} ,文件内容即为本次作业计算结果。
\end{document}
