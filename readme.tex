\documentclass[a4paper,12pt]{ctexart}
\usepackage{xeCJK}
\usepackage[backend = biber, style = gb7714-2015, url=false,gbtitlelink=true]{biblatex}
\usepackage{booktabs}
\usepackage{enumerate}
\usepackage{indentfirst}
\usepackage{graphicx}
\usepackage{longtable}
\usepackage[normalem]{ulem}
\usepackage{amsmath}
\usepackage{amsfonts}
\usepackage{amssymb}
\usepackage{float}
\usepackage{capt-of}
\usepackage[hidelinks]{hyperref}
\usepackage[table,xcdraw]{xcolor}
\author{董晨阳 \thanks{学号1800015446} 许天朗 \thanks{学号1800015428} 李昭伦 \thanks{学号1800015469}}
\date{\today}
\title{风管模型基金运行报告}
\begin{document}
\maketitle
\tableofcontents
\section{风险偏好和目标设定}
作为2021 年年底成立的新基金,本基金被动追踪一篮子股票的价值,持仓份额如表\ref{holdings} 所示。
\begin{table}[ht]
	\begin{tabular}{rrrrrrrrr}
		比亚迪    & 牧原股份    & 东方财富    & 温氏股份   & 宁德时代   \\
		0.0516 & -0.0129 & -0.0957 & 0.2051 & 0.0193 \\
		招商银行   & 中国平安    & 中国建筑    & 中国中免            \\
		0.0521 & 0.3665  & 0.3774  & 0.0366          \\
	\end{tabular}
    \caption{持仓份额}
    \label{holdings}
\end{table}

遵循一般私募基金运行惯例,本基金预警线设置为份额净值 0.8 元,止损线设置为份额净值 0.7 元。
触及预警线时,私募行业一般要求管理人及时减仓,直到净值重归预警线之上才能加仓。
触及止损线时,一般要求管理人无条件清仓,并将剩余的投资款返还客户。
双线作为一把“双刃剑”,一方面明确了安全边际,促使管理人谨慎投资并配置风控预案。另一方面,双线也加大了管理人在极端行情下的操作难度,大规模私募基金的减仓清盘行为甚至会加剧市场的波动。

本基金的业绩基准为沪深300 ETF。因而本基金的目标为在 300ETF 之上取得相对收益的同时,在四个月运行期间避免触及预警和清盘线。综上,本基金的风险偏好为每月最多亏损\(1-0.8^{1/4}\approx 0.05\),即每月净值下跌至多5\%。


\section{策略理由和面临的风险}
因此本基金将采取 protective put 策略,即在持有股票的同时,每月月初在市场上购入适当的月末到期的、执行价格为 95\% 的沪深 300ETF 看跌期权。
\section{当前策略成果}
\end{document}
