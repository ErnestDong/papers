% !TeX root = economics.tex
\documentclass[a4paper,12pt]{article}
\usepackage{xeCJK}
\usepackage{indentfirst}
\usepackage{amsmath}
\usepackage{float}
\usepackage[scale=0.8]{geometry}
\usepackage{fancyhdr}
\pagestyle{fancy}
\usepackage[hidelinks]{hyperref}
\title{经济学综合第一次作业}
\author{董晨阳 2201211201}
\begin{document}
\maketitle
\section{7.E.1}
\subsection{a}

选手1有三个信息集,其策略可以是
\begin{eqnarray}
    S_1=\{&\nonumber\\
    &(L,x,x),(L,x,y),(L,y,x),(L,y,y)\nonumber\\
    &(M,x,x),(M,x,y),(M,y,x),(M,y,y)\nonumber\\
    &(R,x,x),(R,x,y),(R,y,x),(R,y,y)\nonumber\\
    \}&
    \label{eq:1a1}
\end{eqnarray}

式\ref{eq:1a1}中的元素$(a_1,a_2,a_3)$含义为在最顶端信息集采用策略$a_1$,左下信息集策略为$a_2$,右下信息集合策略为$a_3$

选手2只有一个信息集,因此其策略是
\begin{equation}
    S_2=\{ (l),(r) \}
\end{equation}

\subsection{b}\label{sec:1.2}

记选手2选取l、r的策略概率分别为$q_1,q_2$,选手1任意给定的行为策略为:
\begin{enumerate}
    \item 在最顶端信息集选取L、M、R三种策略的概率为$p_{11},p_{12},p_{13}$,
    \item 左下信息集选取x、y的策略概率分别为$p_{21},p_{22}$,
    \item 右下信息集选取x、y的策略概率分别为$p_{31},p_{32}$;
\end{enumerate}
其中 $\sum_j p_{ij}=1,\forall i\in\{ 1,2,3 \}$

则选手1的行为策略终止节点的分布如表\ref{tab:1b1}所示。
\begin{table}[H]
    \centering
    \begin{tabular}{cccccccccc}
          & $T_0$             & $T_1$             & $T_2$             & $T_3$             & $T_4$             \\\hline
        P & $p_{11}$          & $p_{12}q_1p_{21}$ & $p_{12}q_1p_{22}$ & $p_{12}q_2p_{21}$ & $p_{12}q_2p_{22}$ \\\hline
          & $T_5$             & $T_6$             & $T_7$             & $T_8$             &                   \\\hline
        P & $p_{12}q_1p_{31}$ & $p_{12}q_1p_{32}$ & $p_{12}q_2p_{31}$ & $p_{12}q_2p_{32}$ &                   \\\hline
    \end{tabular}
    \caption{终止节点的分布}\label{tab:1b1}
\end{table}

表\ref{tab:1b1}的结果与公式\ref{eq:1b}中的混合策略一致
\begin{equation}
    \begin{cases}
        (L,x,x) & P=p_{11}       \\
        (M,x,x) & P=p_{12}p_{21} \\
        (M,y,x) & P=p_{12}p_{22} \\
        (R,x,x) & P=p_{13}p_{31} \\
        (R,x,y) & P=p_{13}p_{32}
    \end{cases}
    \label{eq:1b}
\end{equation}

\subsection{c}

对于选手1的混合策略,记式\ref{eq:1a1}中策略的概率分别为$p_i,i\in \{ 0,1,\dots,12 \}$,
选手2依然如\ref{sec:1.2}中所示表示,则终止节点的分布为:

\begin{table}[H]
    \centering
    \begin{tabular}{cccccccccc}
          & $T_0$             & $T_1$                & $T_2$             & $T_3$                & $T_4$          \\ \hline
        P & $\sum_1^4p_i$     & $(p_5+p_6)q_1$       & $(p_7+p_8)q_1$    & $(p_5+p_6)q_2$       & $(p_7+p_8)q_2$ \\ \hline
          & $T_5$             & $T_6$                & $T_7$             & $T_8$                &                \\\hline
        P & $(p_9+p_{10})q_1$ & $(p_{11}+p_{12})q_1$ & $(p_9+p_{10})q_2$ & $(p_{11}+p_{12})q_2$ &                \\\hline
    \end{tabular}
    \caption{终止节点的分布}\label{tab:1b2}
\end{table}

等价于:
\begin{enumerate}
    \item 在最顶端信息集选取L、M、R三种策略的概率为$\sum_1^4p_i,\sum_5^8p_i,\sum_9^{12}p_i$,
    \item 左下信息集选取x、y的策略概率分别为$(p_5+p_6)/\sum_{5}^{8}p_i,(p_7+p_8)/\sum_{5}^{8}p_i$,
    \item 右下信息集选取x、y的策略概率分别为$(p_9+p_{10})/\sum_{9}^{12}p_i,(p_{11}+p_{12})/\sum_{9}^{12}p_i$
\end{enumerate}

\subsection{d}

选手1在第二次决策时,没有了第一次决策选择的是M还是R的记忆,因而不再是完美记忆博弈。

\ref{sec:1.2}中的结论依然成立。记第二次决策l、r的策略概率分别为 $p_{21},p_{22}$,终止节点的概率分布为
\begin{table}[H]
    \centering
    \begin{tabular}{cccccccccc}
          & $T_0$             & $T_1$             & $T_2$             & $T_3$             & $T_4$             \\\hline
        P & $p_{11}$          & $p_{12}q_1p_{21}$ & $p_{12}q_1p_{22}$ & $p_{12}q_2p_{21}$ & $p_{12}q_2p_{22}$ \\\hline
          & $T_5$             & $T_6$             & $T_7$             & $T_8$                                 \\\hline
        P & $p_{12}q_1p_{21}$ & $p_{12}q_1p_{22}$ & $p_{12}q_2p_{21}$ & $p_{12}q_2p_{22}$                     \\\hline
    \end{tabular}
    \caption{终止节点的分布}
\end{table}

等价于
\begin{equation}
    \begin{cases}
        (L,x) & P=p_{11}       \\
        (M,x) & P=p_{12}p_{21} \\
        (M,y) & P=p_{12}p_{22} \\
        (R,x) & P=p_{13}p_{21} \\
        (R,y) & P=p_{13}p_{22}
    \end{cases}
\end{equation}

\section{8.B.5}

\subsection{a}\label{sec:2a}

假设选手j生产$q_j$,选手i的最优反应是
$$\max (a-b(q_i+q_j)-c)q_i$$
F.O.C.
$$a-b(2q_i+q_j)-c=0$$
即$q_i=\frac{a-c}{2b}-\frac{q_j}{2}$

当$q_1\ge 0$时,最优反应$q_2\le\frac{a-c}{2b}$;
因而此时 $q_1\ge \frac{a-c}{4b}$,继续推判断依据链可以得到$q_2 \le \frac{3(a-c)}{8b}$。
最终可以得到$q_1=q_2=\frac{a-c}{3b}$,博弈达到唯一结果

\subsection{b}

加入新的选手k,\ref{sec:2a}中的最优反应变为
$$\max (a-b(q_i+q_j+q_k)-c)q_i$$
F.O.C.
$$a-b(2q_i+q_j+q_k)-c=0$$

由于$q_2,q_3\ge 0$,最优反应$q_1\le\frac{a-c}{2b}$,同理$q_3\le\frac{a-c}{2b}$,
继续推判断依据链可以得到此时$q_2\ge 0$,无法进一步推断,因此无法达到唯一的均衡结果。

\section{8.C.2}

记选手i重复删除N轮之后的最优反应策略集为$\Sigma_i^N$。
假设 $s_i$不是$\Sigma_{-i}^N$中任何一个策略的最优反应,并且$s_i$不会在N+1轮删除。
在第 $N+1$轮,$s_i$也不是$\Sigma_{-1}^{N+1}\subseteq \Sigma_{-i}^N$的最优反应,因而在下一轮一定会被删掉。

\end{document}
