% !TeX root = economics.tex
\documentclass[a4paper,12pt]{ctexart}
% \usepackage{xeCJK}
\usepackage{indentfirst}
\usepackage{amsmath}
\usepackage{amsfonts}
\usepackage{float}
\usepackage{enumerate}
\usepackage{diagbox}
\usepackage[scale=0.8]{geometry}
\usepackage[hidelinks]{hyperref}
\title{经济学综合第一次作业}
\author{董晨阳 2201211201}
\begin{document}
\maketitle
\section{8.D.2}

假定存在一个混合策略纳什均衡$(s_1^*,s_2^*,\dots,s_n^*)$,其中 $i$ 的策略被重复删除严格劣势策略第 $k$ 轮所删除。则
\begin{equation}
    \forall s_{-i}\in S_{-i}^{k-1},\exists \sigma_i,a_i\in S_i^{k-1}, \mu_i(\sigma_i,s_{-i})>\mu_i(a_i,s_{-i})
\end{equation}
其中策略 $a_i$ 为以正概率$s_i^*(a_i)$ 被选取的某个纯策略。

又由于$s_{-i}\in S_{-i}^{k-1}$,因此$\mu_i(\sigma_i,s^*_{-i})>\mu_i(a_i,s^*_{-i})$,令$s'$为用$s^*(a)$概率选$\sigma_i$替换$a_i$
从而
\begin{equation}
    \mu(s'_i,s_{-i}^*)=\mu(s_i^*,s^*_{-i})+s^*_i(a_i)(\mu(\sigma_i,s_{-i}^*)-\mu(a_i,s_{-i}^*))>\mu(s_i^*,s_{-i}^*)
\end{equation}

与$(s_1^*,s_2^*,\dots,s_n^*)$是纳什均衡矛盾,因而纳什均衡一定会在重复删除严格劣势策略后幸存。

又因为只留下一个策略组合,根据纳什均衡的存在性,该策略组合一定是一个纳什均衡

\section{8.D.4}

\subsection{a}

假定一个参与人报价大于100,则另一个参与人的任何策略在收益上是等价的,因此不存在一个严格劣势策略

\subsection{b}

\begin{enumerate}
    \item 当参与人i报价大于100时,参与人j的任何策略在收益上都是等价的。
    \item 当参与人 i 报价 $i\in[0,100]$时,参与人j会报价$100-i\le 100$,大于100的策略下收益均小于0。
\end{enumerate}

因此当,报价大于100美元的策略都是弱劣势的。

\subsection{c}

纳什均衡为 $(i,100-i),\forall i\in[0,100]$。因为当参与人i报价为$i$时,
\begin{enumerate}
    \item 如果参与人j报价$j=100-i$,则二人的收益可以满足
    \item 如果参与人j报价$j>100-i$,则$i+j>100$,二人得到的支付为0
    \item 如果参与人j报价$j<100-i$,则二人的收益可以满足,但是少于第一种情形下的支付
\end{enumerate}

\section{8.D.7}

\subsection{a}

假设$\underline{w}_i=u_i(\sigma_i^*,\sigma_{-i}^*)$,则
\begin{eqnarray}
    \underline{v}_i&=&\min_{\sigma_{-i}}\max_{\sigma_i} \mu_i(\sigma_i,\sigma_{-i})\nonumber\\
    &\ge& \min_{\sigma_{-i}}\mu_i(\sigma_i^*,\sigma_{-i})\nonumber\\
    &=& \mu_i(\sigma_i,\sigma_{-i})\nonumber\\
    &=& \underline{w}_i \label{eq0}
\end{eqnarray}

\subsection{b}

假设$(\sigma_i^*,\sigma_{-i}^*)$是零混合策略纳什均衡,那么有
\begin{eqnarray}
    \mu_i(\sigma_i^*,\sigma_{-i}^*)&=&\max_{\sigma_i}\mu_i(\sigma_i,\sigma_{-i}^*)\\
    \mu_{-i}(\sigma_i^*,\sigma_{-i}^*)&=&\max_{\sigma_{-i}}\mu_{-i}(\sigma_i^*,\sigma_{-i})\label{eq1}
\end{eqnarray}
由于是零和博弈,$\mu_i(\sigma_i^*,\sigma_{-i}^*)=-\mu_{-i}(\sigma_i^*,\sigma_{-i}^*)$,(\ref{eq1})式可以改写为
\begin{equation}
    -\mu_i(\sigma_i^*,\sigma_{-i}^*)=\max_{\sigma_{-i}}(-\mu_i(\sigma_i^*,\sigma_{-i}))=-\min_{\sigma_{-i}}\mu_i(\sigma_i^*,\sigma_{-i})
\end{equation}
从而有
\begin{equation}
    \max_{\sigma_i}\mu_i(\sigma_i,\sigma_{-i}^*)=\mu_i(\sigma_i^*,\sigma_{-i}^*)=\min_{\sigma_{-i}}\mu_i(\sigma_i^*,\sigma_{-i})\label{eq_2}
\end{equation}
进一步,(\ref{eq_2})式左侧套$\displaystyle \min_{\sigma_{-i}}$有
\begin{equation}
    \underline{v}_i=\min_{\sigma_{-i}}\max_{\sigma_i}\mu_i(\sigma_i,\sigma_{-i})\le \max_{\sigma_i}\mu_i(\sigma_i,\sigma_{-i}^*)=\mu_i(\sigma_i^*,\sigma_{-i}^*)\label{eq_3}
\end{equation}
(\ref{eq_2})式右侧套$\displaystyle \max_{\sigma_i}$有
\begin{equation}
    \underline{w}_i=\max_{\sigma_i}\min_{\sigma_{-i}}\mu_i(\sigma_i^*,\sigma_{-i})\ge \mu_i(\sigma_i^*,\sigma_{-i}^*)\label{eq_4}
\end{equation}
(\ref{eq_3})(\ref{eq_4})两式联立得
\begin{equation}
    \underline{v}_i\le\underline{w}_i
\end{equation}
结合(\ref{eq0})可知
\begin{equation}
    \underline{v}_i=\underline{w}_i=\mu_i^*\label{eq:res}
\end{equation}

\subsection{c}

根据纳什均衡和零和的性质有
\begin{eqnarray}
    \mu_i(\sigma_i',\sigma_{-i}') \ge\mu_i(\sigma_i'',\sigma_{-i}') =-\mu_{-i}(\sigma_i'',\sigma_{-i}') \ge-\mu_{-i}(\sigma_i'',\sigma_{-i}'') = \mu_i(\sigma_i'',\sigma_{-i} '') \\
    \mu_i(\sigma_i',\sigma_{-i}') =\mu_{-i}(\sigma_i',\sigma_{-i}') \le-\mu_{-i}(\sigma_i',\sigma_{-i}'') =\mu_{i}(\sigma_i',\sigma_{-i}'') \le \mu_i(\sigma_i'',\sigma_{-i} '')
\end{eqnarray}
结合式(\ref{eq:res})可得
\begin{equation}
    \underline{v}_i=\mu_i(\sigma_i'',\sigma_{-i}'')\ge\mu_i(\sigma_i',\sigma_{-i}'')\ge\mu_i(\sigma_i',\sigma_{-i}')\ge\mu_i(\sigma_i'',\sigma_{-i}')\ge\mu_i(\sigma_i'',\sigma_{-i}'')\ge \underline{v}_i
\end{equation}
因此
\begin{equation}
    \underline{v}_i=\mu_i(\sigma_i'',\sigma_{-i}'')=\mu_i(\sigma_i',\sigma_{-i}'')=\mu_i(\sigma_i',\sigma_{-i}')=\mu_i(\sigma_i'',\sigma_{-i}')
\end{equation}
由于$(\sigma_i',\sigma_{-i}')$是一个纳什均衡,因此$\forall \sigma_i,\mu_i(\sigma_i,\sigma_{-i}')\le\mu_i(\sigma_i',\sigma_{-i}')=\mu_i(\sigma_i'',\sigma_{-i}')$,因此$(\sigma_i'',\sigma_{-i}')$也是一个纳什均衡;同理$(\sigma_i',\sigma_{-i}'')$也是一个纳什均衡
\section{8.E.1}
\begin{enumerate}
    \item AA:不管强弱类型都攻击
    \item AN:强类型则攻击,弱类型则不攻击
    \item NA:强类型则不攻击,弱类型则攻击
    \item NN:不管强弱类型都不攻击
\end{enumerate}
payoff 如表\ref{pic:payoff}所示

\begin{table}[H]
    \centering
    \begin{tabular}{|c|c|c|c|c|}
        \hline
        \diagbox{军队2}{军队1} & AA                                                                  & AN                                                                & NA                                                      & NN                            \\\hline
        AA                 & $\displaystyle \frac{M}{4}-\frac{s+w}{2},\frac{M}{4}-\frac{s+w}{2}$ & $\displaystyle \frac{M}{2}-\frac{s+w}{4},\frac{M}{4}-\frac{s}{2}$ & $\displaystyle \frac{3M}{4}-\frac{s+w}{4},-\frac{w}{2}$ & $\displaystyle M,0$           \\\hline
        AN                 & $\displaystyle \frac{M}{4}-\frac{s}{2},\frac{M}{2}-\frac{s+w}{4}$   & $\displaystyle \frac{M-s}{4},\frac{M-s}{4}$                       & $\displaystyle \frac{M}{2}-\frac{s}{4},\frac{M-w}{4}$   & $\displaystyle \frac{M}{2},0$ \\\hline
        NA                 & $\displaystyle -\frac{w}{2},\frac{3M}{4}-\frac{s+w}{4}$             & $\displaystyle \frac{M-w}{4},\frac{M}{2}-\frac{s}{4}$             & $\displaystyle \frac{M-w}{4},\frac{M-w}{4}$             & $\displaystyle \frac{M}{2},0$ \\\hline
        NN                 & $\displaystyle 0,M$                                                 & $\displaystyle 0,\frac{M}{2}$                                     & $\displaystyle 0,\frac{M}{2}$                           & $\displaystyle 0,0$           \\\hline
    \end{tabular}
    \caption{策略组合的期望收益}
    \label{pic:payoff}
\end{table}
这个标准形式博弈的任何纳什均衡都是原博弈的一个贝叶斯纳什均衡。

\begin{enumerate}
    \item 情形1:$M>w>s,w>\frac{M}{2}>s$,从上面的矩阵可以看到$(AA,AN)$和$(AN,AA)$是纯策略贝叶斯纳什均衡。
    \item 情形2:$M>w>s,\frac{M}{2}<s$,从上面的矩阵可以看到$(AA,NN)$和$(NN,AA)$是纯策略贝叶斯纳什均衡。
    \item 情形3:$w>M>s,\frac{M}{2}<s$,从上面的矩阵可以看到$(AA,NN)$,$(NN,AA)$和$(AN,AN)$是纯策略贝叶斯纳什均衡。
    \item 情形4:$w>M>s,\frac{M}{2}>s$,从上面的矩阵可以看到$(AA,AN)$,$(AN,AA)$和$(AN,AN)$F是纯策略贝叶斯纳什均衡。
\end{enumerate}
\end{document}
