\documentclass[a4paper,12pt]{ctexart}
% \usepackage{xeCJK}
\usepackage{csvsimple}
\usepackage{amsmath}
\usepackage{amsfonts}
\usepackage{float}
\usepackage{enumerate}
\usepackage{diagbox}
\usepackage{graphicx}
\usepackage{tikz}
\usetikzlibrary{calc}
\tikzset{
solid node/.style={circle,draw,inner sep=1.5,fill=black},
hollow node/.style={circle,draw,inner sep=1.5}
}
\usepackage[scale=0.8]{geometry}
\usepackage[hidelinks]{hyperref}
\renewcommand{\thesubsection}{\arabic{subsection}}
\title{经济学综合第四次作业}
\author{董晨阳 2201211201}
\begin{document}
\maketitle
\section*{9.C.1}
\subsection{充分性}
当(i)与(ii)同时成立时,在均衡路径上的信息集上,序贯理性意味着每个选手对对手的策略选择了最优反应,并且他们的信念是正确的,因此$\sigma$是一个纳什均衡
\subsection{必要性}
使用反证法假设$\sigma$是一个纳什均衡,而(i)或(ii)不成立。那么
\begin{enumerate}
    \item 如果(i)不成立,则对不成立的均衡路径上的点,该选手对对手的反应不是最优的,与$\sigma$是一个纳什均衡矛盾
    \item 如果(ii)不成立,那么对不成立的均衡路径上的点,该选手的信念在这个点上是不正确的,选手有激励修正自己的信念以获得更大支付,与$\sigma$是一个纳什均衡矛盾
\end{enumerate}
综上所述,当$\sigma$是一个纳什均衡时,(i)和(ii)均成立
\section*{9.C.2}
企业I选择斗争的当且仅当$-1\ge-2u_1+1(1-u_1)$即$\displaystyle u_1\ge \frac{2}{3}$。分类讨论:
\begin{enumerate}
    \item 当$\displaystyle u_1<\frac{2}{3}$时,企业I必定选择容忍。但这样一来企业E一定会选择以策略1进入,即$u_1=1$,矛盾,因此$\displaystyle u_1\ge \frac{2}{3}$。
    \item 当$\displaystyle u_1>\frac{2}{3}$时,企业I必定选择斗争。此时企业E进入的支付最大为$\gamma$小于0,因此企业E选择不进入,支持所有的I的信念,因此企业E选择不进入、企业I选择斗争,同时$u_1>\frac{2}{3}$是一个弱完美贝叶斯均衡。
    \item 当$\displaystyle u_1=\frac{2}{3}$时,企业E的策略一定是以策略1或策略2进入的混合策略,选择策略1的概率是选择策略2的2倍。企业I选择斗争的概率$\sigma_F$应使得E得到的支付在两个纯策略上无差异,即$$-1\sigma_F+3(1-\sigma_F)=\gamma \sigma_F+2(1-\sigma_F)$$即$\displaystyle \sigma_F=\frac{1}{\gamma+2}$,
          企业E的支付为$\displaystyle \frac{3\gamma+2}{\gamma+2}$。此时又可以分三种情况讨论:
          \begin{itemize}
              \item $\gamma<-\frac{2}{3}$时,企业E支付小于0,因此E选择不进入,这个均衡中$(\sigma_0,\sigma_1,\sigma_2)=(1,0,0)$符合I的信念$\mu=\frac{2}{3}$,且I选择斗争的概率为$\sigma_F=\frac{1}{\gamma+2}$
              \item $\gamma>-\frac{2}{3}$时,企业E支付大于0,因此E选择进入,这个均衡中$(\sigma_0,\sigma_1,\sigma_2)=(0,\frac{2}{3},\frac{1}{3})$符合I的信念$\mu=\frac{2}{3}$,且I选择斗争的概率为$\sigma_F=\frac{1}{\gamma+2}$
              \item $\gamma=-\frac{2}{3}$时,企业E支付等于0,E选择进入与否无差异,这个均衡中$(\sigma_0,\sigma_1,\sigma_2)=(1-3x,2x,x),x\in[0,\frac{1}{3}]$符合I的信念$\mu=\frac{2}{3}$,且I选择斗争的概率为$\sigma_F=\frac{1}{\gamma+2}$
          \end{itemize}
\end{enumerate}
\section*{9.C.7}
\subsection*{(a)}
唯一的子博弈完美纳什均衡为选手1选择B,选手2在选手1选B时选D,在选手1选T时选U,记为$\{ B,DU \}$
存在两类纳什均衡:
\begin{enumerate}
    \item 选手1选择B,选手2以$p$的概率选择UU,$1-p$的概率选择DU,需满足$4(1-p)+p\le 2$即$p\ge \frac{2}{3}$
    \item 选手1选择T,选手2以$p$的概率选择DU,$1-p$的概率选择DD,需满足$5(1-p)+2p\le 4$即$p\ge \frac{1}{3}$
\end{enumerate}
\subsection*{(b)}\label{sec:b}
展开形与标准形博弈如图\ref{fig:b1}所示,T是选手1的严格优势策略,因而有唯一的纳什均衡$(T,U)$
\begin{figure}[htbp]
    \label{fig:b1}
    \begin{minipage}{0.48\linewidth}
        \begin{tikzpicture}
            \tikzstyle{level 1}=[level distance=15mm,sibling distance=35mm]
            \tikzstyle{level 2}=[level distance=15mm,sibling distance=15mm]
            \node(0)[solid node,label=above:{选手1}]{}
            child{node(1)[solid node]{}
            child{node[hollow node,label=below:{$(4,2)$}]{} edge from parent node[left]{$D$}}
            child{node[hollow node,label=below:{$(1,1)$}]{} edge from parent node[right]{$U$}}
            edge from parent node[left,xshift=-3]{$B$}
            }
            child{node(2)[solid node]{}
            child{node[hollow node,label=below:{$(5,1)$}]{} edge from parent node[left]{$D$}}
            child{node[hollow node,label=below:{$(2,2)$}]{} edge from parent node[right]{$U$}}
            edge from parent node[right,xshift=3]{$T$}
            };
            \draw[dashed,rounded corners=10]($(1) + (-.2,.25)$)rectangle($(2) +(.2,-.25)$);
            \node at ($(1)!.5!(2)$) {选手2};
        \end{tikzpicture}
    \end{minipage}
    \begin{minipage}{0.48\linewidth}
        \begin{tabular}{|r|c|c|}
            \hline
            \diagbox{选手1}{选手2} & D   & U   \\\hline
            T                  & 5,1 & 2,2 \\\hline
            B                  & 4,2 & 1,1 \\\hline
        \end{tabular}
    \end{minipage}
    \caption{\nameref{sec:b}展开形与标准形博弈}
\end{figure}

\subsection*{(c)}\label{sec:c}
展开形博弈如图\ref{fig:c1}所示
\begin{figure}[htbp]
    \label{fig:c1}
    \centering
    \begin{tikzpicture}[scale=2]
        \tikzstyle{level 1}=[level distance=15mm,sibling distance=45mm]
        \tikzstyle{level 2}=[level distance=15mm,sibling distance=20mm]
        \tikzstyle{level 3}=[level distance=15mm,sibling distance=12mm]
        \node(0)[solid node,label=above:{自然}]{}
        child{
            node(1)[solid node]{}
            child{
                node(3)[hollow node]{}
                child{
                    node[hollow node,label=below:{$(4,2)$}]{}
                    edge from parent node[left]{$D$}
                }
                child{
                    node[hollow node,label=below:{$(1,1)$}]{}
                    edge from parent node[right]{$U$}
                }
                edge from parent node[left]{$B$}
            }
            child{
                node(4)[solid node]{}
                child{
                    node[hollow node,label=below:{$(5,1)$}]{}
                    edge from parent node[left]{$D$}
                }
                child{
                    node[hollow node,label=below:{$(2,2)$}]{}
                    edge from parent node[right]{$U$}
                }
                edge from parent node[right]{$T$}
            }
            edge from parent node[left,xshift=-3]{$p$}
        }
        child{
            node(2)[solid node]{}
            child{
                node(5)[solid node]{}
                child{
                    node[hollow node,label=below:{$(4,2)$}]{}
                    edge from parent node[left]{$D$}
                }
                child{
                    node[hollow node,label=below:{$(1,1)$}]{}
                    edge from parent node[right]{$U$}
                }
                edge from parent node[left]{$B$}
            }
            child{
                node(6)[hollow node]{}
                child{
                    node[hollow node,label=below:{$(5,1)$}]{}
                    edge from parent node[left]{$D$}
                }
                child{
                    node[hollow node,label=below:{$(2,2)$}]{}
                    edge from parent node[right]{$U$}
                }
                edge from parent node[right]{$T$}
            }
            edge from parent node[right,xshift=-3]{$1-p$}
        };
        \draw[dashed,rounded corners=10]($(1) + (-.2,.25)$)rectangle($(2) +(.2,-.25)$);
        \draw[dashed,rounded corners=10]($(4) + (-.2,.25)$)rectangle($(5) +(.2,-.25)$);
        \draw[dashed,rounded corners=10]
        ($(3) + (-.2,.25)$)--($(3) + (.2,.25)$)--
        ($(4) + (0,.80)$)--($(5) + (0,.80)$)--
        ($(6) + (-.2,.25)$)--($(6) + (.2,.25)$)--
        ($(6) + (.2,-.25)$)--($(6) + (-.2,-.25)$)--
        ($(5) + (0,.30)$)--($(4) + (0,.30)$)--
        ($(3) + (.2,.-.25)$)--($(3) + (-.2,-.25)$)--
        cycle;
        \node at ($(1)!.5!(2)$) {选手1};
        \node at ($(4)!.5!(5)$) {选手2$I_T$};
        \node[above,yshift=25] at ($(3)!.5!(6)$) {选手2$I_B$};
    \end{tikzpicture}
    \caption{\nameref{sec:c}展开形博弈}
\end{figure}


记$r$是选手2位于信息集$I_B$时选手1选择$B$的条件概率,$q$是选手2位于信息集$I_T$时选手1选择$B$的条件概率,$s$为选手1选择$B$的概率,对$s$分类讨论有:
\begin{enumerate}
    \item $s=0$时是WBPE,此时选手1永远选择$T$,而选手2观察到的$q=r=0$,将永远选择$U$,符合I的偏好,因而是WBPE
    \item $s=1$时不可能是WBPE,若选手1一定选择$B$,则选手2一定选择$D$。而\nameref{sec:b}中结论有$T$是占优策略,因此选手1选择$T$支付更大,所以不是均衡
    \item $s\in(0,1)$时,选手2选择U与D的支付分别如表\ref{tab:c1}所示。根据贝叶斯法则,均衡时
    \begin{equation}
        r=\frac{sp}{(1-s)(1-p)+sp}\label{eq:c_1}
    \end{equation}
    \begin{equation}
        q=\frac{s(1-p)}{s(1-p)+p(1-s)}\label{eq:c_2}
    \end{equation}
    此时可以进一步分类为以下四种情况:
    \begin{itemize}
        \item $s>p$且$s>1-p$,此时$q>\frac{1}{2}$,$r>\frac{1}{2}$且$s>\frac{1}{2}$,从表\ref{tab:c1}中可以看出选手2选D的收益更大,而此时选手1选T时占优策略即$s=0$,矛盾因而不存在WPBE
        \item $s<p$且$s<1-p$,此时$q<\frac{1}{2}$,$r<\frac{1}{2}$且$s<\frac{1}{2}$,从表\ref{tab:c1}中可以看出选手2选U的收益更大,而此时选手1选T时占优策略即$s=0$,与$s\in(0,1)$矛盾因而不存在WPBE
        \item $1-p<s<p$,此时$q<\frac{1}{2}$,$r>\frac{1}{2}$且$p>\frac{1}{2}$,从表\ref{tab:c1}中可以看出选手2在$I_T$中选择$U$,$I_B$中选择$D$;而对选手1,B的期望支付为$4p+1(1-p)$,A的期望支付为$2p+5(1-p)$,因此$p=\frac{2}{3}$时有WBPE,此时选手1以$s\in(\frac{1}{3},\frac{2}{3})$选择B,$1-s$概率选择T;选手2在选手1选B时选D,$I_B$中选择D
        \item $p<s<1-p$,此时$q>\frac{1}{2}$,$r<\frac{1}{2}$且$p<\frac{1}{2}$,从表\ref{tab:c1}中可以看出选手2在$I_T$中选择$D$,$I_B$中选择$U$;类比上一条有$p=\frac{1}{3}$时有WBPE,此时选手1以$s\in(\frac{1}{3},\frac{2}{3})$选择B,$1-s$概率选择T;选手2在选手1选B时选D,$I_B$中选择D
    \end{itemize}
\end{enumerate}
\begin{table}[htbp]
    \centering
    \begin{tabular}{|r|c|c|}
        \hline
        \diagbox{所在信息集}{行动}&$I_B$&$I_T$\\\hline
        U&$2-r$&$2-q$\\\hline
        D&$1+r$&$1+q$\\\hline
    \end{tabular}
    \label{tab:c1}
    \caption{选手2的payoff}
\end{table}
综上存在唯一的纯策略WBPE,当$p\in \{ \frac{1}{3},\frac{2}{3} \}$时,存在混合策略的WPBE
\end{document}
