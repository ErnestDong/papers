% !TeX root = economics.tex
\documentclass[a4paper,12pt]{ctexart}
% \usepackage{xeCJK}
\usepackage{indentfirst}
\usepackage{amsmath}
\usepackage{amsfonts}
\usepackage{float}
\usepackage{enumerate}
\usepackage{diagbox}
\usepackage{graphicx}
\usepackage{tikz}
% https://www.sfu.ca/~haiyunc/notes/Game_Trees_with_TikZ.pdf
\usetikzlibrary{calc}
\usepackage[scale=0.8]{geometry}
\usepackage[hidelinks]{hyperref}
\title{经济学综合第三次作业}
\author{董晨阳 2201211201}
\begin{document}
\maketitle
\section*{9.B.3}
令选手1的纯策略为$s_1\in \{ L,R \}$,选手2的策略为$s_2\{ a,b \}$,选手三策略$s_3\in \{ x,y,z \}$,其中$x,y,z$分别是选手1选择$L$的信息集、选手1选择$R$选手2选择$a$的信息集、选手1选择$R$选手2选择$b$的信息集。
在这个例子中,纯策略子博弈完美纳什均衡为$(R,a,(r,r,l))$,这个策略是纳什均衡,逐个验证可知:
\begin{enumerate}
    \item 如果选手1选择$s_1=L$则选手1的支付为$-1<5$
    \item 如果选手2选择$s_2=b$则选手2的支付为$-1<4$
    \item 如果选手3选择$s_3=\{ x,r,z \}$则选手3的支付为$2<4$
\end{enumerate}

博弈一共有四个纯策略纳什均衡,分别为$(s_1,s_2,s_3)=(R,a,(x,r,z)),\{ x,y \}\subseteq \{ r,l \} $,产生了相同的结果。
但是仅有$(R,a,(r,r,l))$满足序贯理性,其他三个结果中选手3的行动在某个节点非理性,逐个验证可知:
\begin{enumerate}
    \item 若$s_3=(l,r,z)$,即当选手1选择$L$时选手3选择$l$,此时若选择$r$得到的支付更高,不满足序贯理性
    \item 若$s_3=(r,r,r)$,即当选手1选择$R$选手2选择$b$时选手3选择$r$,此时若选择$l$得到的支付更高,不满足序贯理性
\end{enumerate}

\section*{9.B.5}
\subsection*{(a)}\label{subsec:a}
每个选手有$m_i$个纯策略;
如果允许混合策略,选手i的策略为$$\Sigma_i=\{ (p_1,\dots,p_{m_i} )\in \mathbb{R}_{+}^{m_i}:p_k\ge0,\forall k|\sum_{k=1}^{m_i}p_k=1 \}$$

\subsection*{(b)}
选手1行动仍然有$m_1$个纯策略,但选手2可以根据选手1的行动相机抉择,因此有$m_2^{m_1}$个纯策略

\subsection*{(c)}
假设对于$\forall (m_1,m_2)(m_1',m_2'),\phi_i(m_1,m_2)\neq \phi_i(m_1',m_2')$,那么选手2在每个节点上支付都不一样,因此每个节点上有唯一的最优反应。
通过逆向归纳,并且选手1也缺乏无差异性,因此选手1对选手2的子博弈也存在唯一的最优反应,与存在多个子博弈完美纳什均衡矛盾。因此$\exists  (m_1,m_2)(m_1',m_2'),\phi_i(m_1,m_2)= \phi_i(m_1',m_2')$

\subsection*{(d)}
令$(m_1^*,m_2^*)$为\nameref{subsec:a}中的纳什均衡,选手1的支付为$\pi_1$。
那么选手1选择$m_1^*$,选手2无论如何都选择$m_{2}^*$是一个纳什均衡但不是一个子博弈完美纳什均衡。
SPNE一定有如果选手1选择$m_1^*$那么选手2选择$m_2^*$,因此选手1的选$m_{1}^*$后可以获得$\pi_1$,而如果存在其他子博弈可以得到更多,则选手一的支付一定不小于$\pi_1$

这个结论对序贯博弈的纳什均衡不一定成立,如果选手2进行了可信的威胁,那么在纳什均衡中,选手1的支付可能更低。
\subsection*{(e)}
\subsubsection*{(1)}
% Node styles
\tikzset{
    % Two node styles for game trees: solid and hollow
    solid node/.style={circle,draw,inner sep=1.5,fill=black},
    hollow node/.style={circle,draw,inner sep=1.5}
}
\begin{minipage}[H]{0.4\linewidth}
    \begin{table}[H]
        \begin{tabular}{|r|c|c|}
            \hline
            \diagbox{选手1}{选手2} & L       & R       \\\hline
            U                  & $(0,1)$ & $(1,1)$ \\\hline
            D                  & $(0,0)$ & $(1,0)$ \\\hline
        \end{tabular}
    \end{table}
\end{minipage}
\begin{minipage}[H]{0.4\linewidth}
    \begin{tikzpicture}[]
        \tikzstyle{level 1}=[level distance=15mm,sibling distance=35mm]
        \tikzstyle{level 2}=[level distance=15mm,sibling distance=15mm]
        \node(0)[solid node,label=above:{选手1}]{}
        child{
        node(1)[solid node,label=right:{选手2}]{}
        child{node[hollow node,label=below:{$(0,1)$}]{} edge from parent node[left]{$L$}}
        child{node[hollow node,label=below:{$(1,1)$}]{} edge from parent node[right]{$R$}}
        edge from parent node[left,xshift=-3]{$U$}
        }
        child{
        node(2)[solid node,label=right:{选手2}]{}
        child{node[hollow node,label=below:{$(0,0)$}]{} edge from parent node[left]{$L$}}
        child{node[hollow node,label=below:{$(1,0)$}]{} edge from parent node[right]{$R$}}
        edge from parent node[right,xshift=3]{$D$}
        };
    \end{tikzpicture}
\end{minipage}

这个博弈中条件(ii)成立,但任何路径都是子博弈完美纳什均衡,而标准型的均衡只有$(U,R)$,支付为$(1,1)$

\subsubsection*{(2)}
\begin{minipage}[H]{0.4\linewidth}
    \begin{table}[H]
        \begin{tabular}{|r|c|c|}
            \hline
            \diagbox{选手1}{选手2} & L       & R       \\\hline
            U                  & $(1,0)$ & $(0,1)$ \\\hline
            D                  & $(0,1)$ & $(1,0)$ \\\hline
        \end{tabular}
    \end{table}
\end{minipage}
\begin{minipage}[H]{0.4\linewidth}
    \begin{tikzpicture}[]
        \tikzstyle{level 1}=[level distance=15mm,sibling distance=35mm]
        \tikzstyle{level 2}=[level distance=15mm,sibling distance=15mm]
        \node(0)[solid node,label=above:{选手1}]{}
        child{
        node(1)[solid node,label=right:{选手2}]{}
        child{node[hollow node,label=below:{$(1,0)$}]{} edge from parent node[left]{$L$}}
        child{node[hollow node,label=below:{$(0,1)$}]{} edge from parent node[right]{$R$}}
        edge from parent node[left,xshift=-3]{$U$}
        }
        child{
        node(2)[solid node,label=right:{选手2}]{}
        child{node[hollow node,label=below:{$(0,1)$}]{} edge from parent node[left]{$L$}}
        child{node[hollow node,label=below:{$(1,0)$}]{} edge from parent node[right]{$R$}}
        edge from parent node[right,xshift=3]{$D$}
        };
    \end{tikzpicture}
\end{minipage}

这个博弈中,混合策略纳什均衡带来的期望收益为$\displaystyle \frac{1}{2}$,但子博弈纳什均衡仅有两个$(U,(R,L))$和$(D,(L,R))$,选手1的收益只能是0
\section*{9.B.6}
假设企业E进入后以$x$概率选择小利基,企业I以$y$概率选择大利基。为了让企业I选择大小利基无差异,需要有
$$
-6x-1(1-x)=1x-3(1-x)
$$
同理为了让企业E选择大小利基无差异,需要有
$$
-6y-1(1-y)=1y-3(1-y)
$$
从而得到
$$
x=\frac{2}{9},y=\frac{2}{9}
$$
此时企业E的支付为$\displaystyle 1x-3(1-x)=-\frac{19}{9}$,因此企业E的决定是不进入市场,所以子博弈完美纳什均衡为:
\begin{itemize}
    \item 企业E在第一个节点不进入,在第二个节点上$\displaystyle \frac{2}{9}$的概率选择小利基,$\displaystyle \frac{1}{5}$的概率选择小利基
    \item 企业I在第二个节点上$\displaystyle \frac{2}{9}$的概率选择小利基,$\displaystyle \frac{2}{9}$的概率选择小利基
\end{itemize}
\end{document}
