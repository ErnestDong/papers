\documentclass{ctexart}
\author{董晨阳 2201211201}
\title{经济建设新探索:中国特色估值体系}
\date{\today}
\begin{document}
\maketitle
\section*{引言}
11月21日,中国证监会主席易会满在2022金融街论坛年会上提出,探索建立具有中国特色的估值体系,促进市场资源配置功能更好发挥。这一重磅发言让“中国特色估值体系”引发市场高度关注,相关中字头股票等应声大涨。究竟如何看待“中国特色估值体系”?目前估值体系是如何构造?这值得我们思考。成熟完善的估值体系对于中国资本市场健康发展和发挥功能有重要意义。在迈向中国式现代化的道路上,需要发展建立中国特色现代资本市场。建立具有中国特色的估值体系,是“建立中国特色资本市场”的重要组成部分。成熟完善的估值体系对于资本市场的价值发现以及资源配置功能的发挥有重要作用,也是资本市场支持实体经济发展的重要基础。

易主席特别指出“探索建立具有中国特色的估值体系,促进市场资源配置功能更好发挥”,应“深刻认识我们的市场体制机制、行业产业结构、主体持续发展能力所体现的鲜明中国元素、发展阶段特征,深入研究成熟市场估值理论的适用场景,把握好不同类型上市公司的估值逻辑”,为资本市场指出了关键的方向。
\section*{何谓“中国特色估值体系”}

\subsection*{背景}
中国市场当前所处的环境需要中国特色估值体系的支持。当前中国的经济体量已稳居全球第二,中国资本市场也是全球第二大资本市场,2022年初至今,上交所和深交所的IPO募资规模已经达到全球第一和第二,成为全球最重要的市场之一。过去我国资本市场建设借鉴学习了不少海外成熟市场的经验,新时代要求中国特色资本市场建设更要形成“学为己用、和而不同”的认知转变,作为资本市场建设的重要一环,估值体系也要适应中国当前的发展环境。近年来,资本市场不断优化升级,金融改革节奏加快,政策催促市场结构优化,A股国际化进程加快,引领一系列经济和企业制度的变革,总体来说船行稳健,成效显著。目前A股市场在全球资本市场上具有举足轻重的地位,这也是资本市场稳健成长、高效发展的成果。

合理的估值体系意义是多方面的。股票估值合理有利于投资者建立长期预期,通过承担一定的风险和波动,获得长期稳定及合理的资本回报,从而把股票作为一类重要的投资资产进行长期配置;合理的估值体系可以促进资本市场的融资和再融资功能有序进行,提高资源配置效率,实现金融和实体经济的良性互促。

资本市场在估值结构、投资者结构和投资者理念方面有显著的发展,也仍有一定提升空间。资管机构参与其中,有责任持续吸引中长期资金入市,鼓励培养长期价值的投资理念,推动中国金融结构转型,提高直接融资比重,“把资本市场一般规律与中国市场的实际相结合、与中华优秀传统文化相结合”,向成熟国际市场取其长、补其短,在打造具备中国特色的投资文化、建立具有中国特色的估值体系方面,发挥机构投资者的积极作用。

\subsection*{为何需要中国特色估值体系}
目前A股市场的确存在一定的估值分化和扭曲,股票定价的不合理会有几个方面的负面影响,第一是影响股票市场资源配置的效率,第二是影响投资者回报的稳定性,进而使得投资者的行为短期化,从而反过来使得估值定价更不合理,形成一个负反馈。对比包括中国在内的各国股票市场最近这几十年的运行态势,可以看到各国的经济运行和上市公司的盈利总体而言都是比较平稳的,因此合理的股票估值,就是市场的定价和波动更能反映基本面的运行特征。股票的平稳性表现为整体的估值波动区间更聚焦于长期均值,体现在年度收益的分布上更聚集于长期均值,体现在市场的结构分化基股票定价更聚焦于公司基本面贡献的长期股东回报。

这是资本市场贯彻20大会议精神,金融支持实业,加强现代化建设的必由之路。资本市场的价值发现需要中国特色的估值体系,特别是在中国的投资者占比95\%以上的前提下,更应该如此。中国资本市场的完善程度不够是中国估值体系尚未建立起来的重要原因,随着市场的完善,符合国际惯例又具备中国特色的估值体系终将建立起来,这是市场发挥配置资源功能的基础工作。

\section*{中国特色社会主义的关键}
\subsection*{专业的机构投资者}
建设中国特色现代资本市场,全面实施注册制改革、健全多层次资本市场体系的过程中,建立“中国特色估值体系”是重要的一环。其中,专业机构投资者将扮演重要的角色。

一方面,我国资本市场是多层次的,根据企业所处发展阶段,融资需求,适用不同估值方法和估值水平,同时在实现全面注册制过程中,也需要建立适合中国上市企业的定价体系。探索中国特色的估值体系,需要研究成熟市场估值理论,选择合适的估值模型。在这过程中机构投资者可以通过发挥专业研究定价能力,促进市场化定价机制有效发挥作用,助力企业价值发现,有利于市场资源配置效率的提升,发挥资本市场支持实体经济的功能。

另一方面,当前A股市场投资者结构中,个人投资者占比仍偏高,给予中短期公司成长性和小市值股价弹性权重过高,而给予公司长期稳健经营、盈利稳定性权重偏低,因此导致部分经营和现金流稳健、高分红的企业估值长期被低估。

此外,长期以来,定价估值体系的不尽完善让很多央企国企估值过低,部分公司股价甚至长期低于净资产,这也变相造成了国有资产的流失。因此需要进一步引入中长期投资者,提升机构投资者比例,并在专业机构投资者引导下建立具有中国特色的成熟的估值体系,从而使A股能够市场长期健康稳定发展。

探索建立具有中国特色的估值体系,促进市场资源配置功能更好发挥,这在中国资本市场是重要且现实的问题。从A股市场的估值体系来看,2016年来外资持续流入A股,与此同时,国内机构投资者的占比也有显著提升,带来一轮A股的核心资产价值重估。而当前,国有资产上市公司估值依然处于较低位,在未来中国特色估值体系下,同样有望迎来重估的机会。

\subsection*{资本市场客观规律}

易主席的主题演讲为当前和今后一段时期证券业全面贯彻落实二十大精神,服务好资本市场改革发展指明了方向和道路。二十大报告要求以“中国式现代化”推进全面建成社会主义现代化强国目标,易主席在主题演讲中也强调,我国资本市场走出了一条既遵循资本市场一般规律,又具有中国特色的发展之路。个人理解,中国式现代化道路下,中国经济发展有其自身特点,国有经济在关键产业和保民生领域等会发挥更加重要的作用,这一点可能需要资本市场在定价中予以充分考虑。

探索建立具有中国特色的估值体系,促进市场资源配置功能更好发挥,这在中国资本市场是重要且现实的问题,方向、结果是“提高直接融资比重”,而估值高低决定了融资功能是否能够很好的实现。当前国有企业估值总体偏低,因此现在监管层领导提出,需要“中国特色”和“资本市场一般规律”的有机融合,要把握好不同情况下上市公司的估值逻辑,要重新理解和公正给予国有企业更高估值。

尽管A股国际化程度一直在提升,但国内资金仍然占据市场定价权,A股的估值更多由内部的基本面预期变化(逻辑)、流动性和风险溢价等因素决定。A股的历史估值中枢稳定,当前处于历史底部区域,讲好中国故事,基本面持续兑现,A股的长期估值中枢具备提升空间。


\subsection*{中国式现代化的精神内核}
中国特色估值体系,应该是建立在社会主义价值观之上的估值体系,符合中国式现代化五大特征的企业应该享有高估值。

在中国式现代化框架下,随着中国经济发展将需要更多兼顾安全与发展,在未来五年,保证国家安全、完善分配制度、健全社会保障体系将成为我国经济发展中的重要发展目标。在多重目标下,中国特色的估值体系,即在投资定价过程中,除了要考虑上市公司的收益性、流动性、平等性等市场化的估值视角,还需结合上市公司发展的稳定性、可持续性、社会责任等中长期经营管理的估值视角,按照投资者结构的实际情况去甄别。

\section*{“中国特色估值体系”的意义}
\subsection*{引入长期投资者、改善投资者结构}
当前A股市场“散户化”特征明显。从投资者结构看,上交所数据显示,截至2020年末,A股市场机构持股占比仅18\%,约20\%的散户贡献了约60\%左右的成交量。

从机构行为看,在短考核久期下,国内公募基金换手率较全球明显偏高,万得数据显示,2021年国内偏股混合型基金换手率中位数为281\%,普通股票型基金换手率中位数为312\%,远高于全球权益共同基金27\%的平均换手率。在此背景下,A股市场波动率明显偏高,估值在泡沫与极度低估之间周期往复,这对A股市场长期发展并无益处。

因此,A股市场应当引入更多长期资金。一方面可以结合相关制度建设,近期个人养老金第三支柱的设立就有助于提升新型长期投资者入市,建议后续可以考虑从提升长期专业投资者在权益市场的配置比例、给予长期投资者一定的税费优势、鼓励公募私募等金融机构业绩评估长期化等角度,提升A股市场长期投资者占比。

另一方面,建议继续积极优化A股市场生态,推动上市公司质量,尤其是银行和国有上市企业质量的进一步提高,包括对上市制度的改革和优化,吸引更多优质公司在A股上市;考虑鼓励大型国有企业尝试分拆上市等各种形式,进一步提升整体市值规模;鼓励银行和国有上市企业结合资本市场环境,进一步优化公司治理,提升运营效率;加大对违规造假行为的惩罚力度,提高违规行为成本,并积极引入科技手段提升监管和违规检查效率,引导上市公司提升信息披露质量。

市场投资者结构确实对中国资本市场估值体系产生一定影响。未来在“中国特色现代资本市场”的积极建设过程中,投资者有望逐步改善对相关板块和上市公司的价值认知。

近年来,随着资本市场不断对外开放,各种投资者开始进入,国内的社保、年金、职业年金等,境外也有很多不同的机构投资者,开始加入到中国的资本市场,机构投资者占比在不断地提升,相比许多年前,现在市场投资者行为更加机构化。机构化的特征反映在市场上,一是市场更稳定,抗压能力也更强了;二是市场的有效性提高了,股价被认可之后的兑现速度更快了。

机构化进程不断提速,但仍有较大的发展空间。根据中金公司测算,截至2022年上半年,机构投资者在A股自由流通市值的占比已经提升至57.9\%,中长期机构投资者的绝对规模和占比仍有进一步提升空间。

需要注意的是,当前A股个人投资者数量已超过2亿,个人投资者的交易占比仍在60\%左右。这意味着资管机构需要更专业、更长期、更注重价值挖掘,在A股市场的估值结构优化、挖掘上市公司价值、保护个人投资者利益方面发挥重要的作用,“把资本市场一般规律与中国市场的实际相结合、与中华优秀传统文化相结合”,形成具备中国特色的投资文化,积极助力建设中国特色的估值体系。

国际成熟的估值体系框架已经成熟,既有理论支持,也有实践检验,成熟的投资者已经对此驾轻就熟。不成熟的投资者在市场的历练中将会逐渐成熟,成熟主要变现为长期、专业和价值发现,这对中国估值体系的完善具有重要的作用。我认为这是一个相对长期的过程,投资者的成熟与市场的成熟互为因果。

当前A股个人投资者数量已超过2亿,个人投资者的交易占比仍在60\%左右。散户居多是A股的资金结构特征,是市场相对容易出现“炒小”、“炒新”的原因。引入更多长期、专业投资者,有望完善中国资本市场估值体系:一是通过政策鼓励社保、险资、外资等天然长线资金入市;二是需要积极倡导价值投资、长期投资理念,培养价值投资环境。

长期资金做资产配置的时候会倾向于给稳定分红的上市公司相应的配置比例,这是长期资金投资期限、投资目标等决定的。中长期资金占比提升,对于稳定市场估值会发挥积极作用。吸引更多中长期投资者、专业投资者需要一个过程,一是持续提升机构专业水平,二是加强投资者教育、普及价值投资理念,三是完善个人养老金和企业年金等税收优惠制度、吸引更多中长期资金入市。

从全球看,各国资本市场投资者结构差异较大,也没有最佳的投资者结构,但机构化是一个大趋势,近年来A股市场机构投资者持股和交易占比稳步上升。同时,积极培养专业化程度较高、注重价值和长期投资的各类金融机构,限制喜欢追逐市场热点、“散户化”特征明显的机构,有助于建设中国特色的估值体系。
\subsection*{营造出追求价值、长期的正确投资价值观}
对于监管层而言,建议继续引导社保、养老金等长期资金入市,发挥长期资金的引导作用,在居民财富资产配置浪潮下,鼓励居民从直接持股转为间接持股,加大投资者教育,通过更为专业的机构投资者参与资本市场。

对于机构投资者而言,在绩效考核和体系中,应进一步鼓励和引导对长期业绩的考核权重,避免机构投资者决策行为的短期化,减少投资中追热点、炒热点的行为,牢固树立长期投资、价值投资理念;应当厚植投资文化理念,鼓励机构投资者积极承担价值发现者和市场稳定器的角色,探索建立具有中国特色的估值体系,促进市场资源配置功能更好发挥。

对于个人投资者而言,应当在投资者教育中融入厚积薄发、久久为功等中华优秀传统文化,培育更加成熟的投资理念,帮助投资者摈弃一夜暴富或“赚快钱”的想法。

首先,倡导理性投资,投资者可以依据自己的财务状况,订立适当的投资目标。其次,提升金融认知,引导投资者熟悉金融市场的基础知识,了解各类投资品的属性,选择合适的投资工具。最后,引导价值投资,通过持续的投资者教育,提高投资者信息收集判断能力,合理使用市场信息,把握好风险和收益关系。

上市公司锐意进取,信息公开,监管有力,市场会形成正确的投资价值观。当然,投资者教育也非常重要,会促进价值观的形成。

冰冻三尺非一日之寒。发达国家几百年资本市场的发展历程中,投资者等市场参与方相对比较成熟,我国资本市场起步晚,还处于持续完善中。目前来看,市场能够做的主要是两件事:一是加强投资者教育;二是完善专业机构考核制度,增大中长期业绩考核权重。

应该以长期考核激励引导市场营造出追求价值、长期的正确投资价值观。严格禁止短期考核和过度激励,建立基金从业人员和基金份额持有人利益绑定机制。董事会对经理层的考核,应当关注基金长期投资业绩、公司合规和风险管理等保护基金份额持有人利益的情况,不得以短期的基金管理规模、盈利增长等作为主要考核标准。

探索建立具有中国特色的估值体系,促进市场资源配置功能更好发挥,这在中国资本市场是重要且现实的问题。在上市公司结构与估值问题上,讲话中既提及中国资本市场“多种所有制经济并存、覆盖全部行业大类、大中小企业共同发展的上市公司结构”特征,又指出应“深刻认识我们的市场体制机制、行业产业结构、主体持续发展能力所体现的鲜明中国元素、发展阶段特征,深入研究成熟市场估值理论的适用场景,把握好不同类型上市公司的估值逻辑”。从整体市场尤其是A股市场的估值体系来看,尽管A股国际化程度近年有所提升,但国内资金仍然主导市场定价,A股的估值更多由内部的基本面预期变化、流动性和风险溢价等因素决定。
\end{document}
