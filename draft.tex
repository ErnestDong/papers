\documentclass{ctexart}
\usepackage{xeCJK}
\usepackage[hidelinks]{hyperref}
\usepackage[backend=biber,style=gb7714-2015ay]{biblatex}
\usepackage{graphicx}
\usepackage{longtable}
\usepackage{wrapfig}
\usepackage{rotating}
\usepackage[normalem]{ulem}
\usepackage{amsmath}
\usepackage{amssymb}
\usepackage{capt-of}
\usepackage[export]{adjustbox} 
\addbibresource{reference.bib}
\title{为何毛泽东思想在年轻人中再次流行}
\author{董晨阳}
\date{\today}
\begin{document}
\maketitle
\tableofcontents
\section{引言}
\subsection{建国以来毛泽东思想的流行趋势}
在21世纪前,毛泽东思想的流行趋势可以用毛选的发行数来表征。

\citeauthor{张慎趋2006毛泽东选集}在研究中发现,
自1951年10月《毛选》第一卷发行至到1965年末,
全国累计印制《毛选》1000多万套。
而文革10年间对毛泽东思想的崇拜使得《毛选》非常热门,
《毛选》印刷量达2.4亿套
(包括少数民族文版、盲文版、外文版)。


文革之后一方面旧有印书量积压,
另一方面由于各种西方思潮影响,
到20世纪80年代,《毛选》在书店无人问津。
毛泽东思想仿佛被人遗忘在角落。
这种情况直至1989年才有所改善。
大学生们在历经纷繁复杂的思潮影响后,
开始寻找《毛选》阅读,呈现“寻找毛泽东热”,
《毛选》又一度脱销。

进入90年代,1991年中央决定出版《毛选》第一卷至第四卷第二版,
一年内全国发行1194万套。

进入21世纪后,纸质传媒工具的影响力式微。
毛泽东思想的流行程度经历了一个先下降后上升的趋势。
加入世贸组织后更多的新思想、新现象涌入中国,
部分媒体和“公共知识分子”对我国社会现象进行了“反思”与“重构”,
社会主义的科学社会主义本质的奇谈怪论,一直不绝于耳,在党内和社会上造成严重思想混乱。

党的十八大以来,
以习近平同志为核心的党中央以巨大的政治勇气和强烈的责任担当,
统揽伟大斗争、伟大工程、伟大事业、伟大梦想,
提出一系列新理念新思想新战略,出台一系列重大方针政策,
% 推出一系列重大举措,推进一系列重大工作,
% 解决了许多长期想解决而没有解决的难题,
% 办成了许多过去想办而没有办成的大事,
推动党和国家事业取得历史性成就、发生历史性变革,
调正了中国巨轮的航向。
毛泽东思想又逐渐重新流行起来。疫情以来流行程度更甚。
我们统计了年轻人聚集的网络问答平台知乎上“就中国而言,近代到今最伟大的人是谁?”的18000余条回答,
部分统计结果如图\ref{zhihu}所示。

\begin{figure}
    \includegraphics[center]{lib/zhihu.png}
    \caption{就中国而言,近代到今最伟大的人是谁?}
    \label{zhihu}
\end{figure}


\subsection{界定“毛泽东思想”}
这里需要特别说明的是,本文使用的“毛泽东思想”概念,
是经中共中央第二个历史决议所重新界定了的概念,
并不是利用所谓“毛泽东思想”为文革招魂。

第二个历史决议定义的概念使得毛泽东思想同毛泽东同志的晚年错误从根本上划清了界限,
从而为在改革开放新的历史条件下继续坚持和发展毛泽东思想、
开创中国特色社会主义打开了指导思想上的通道。
% 即
% \begin{itemize}
%     \item 第一,毛泽东同志和中国共产党,依据新民主主义革命胜利所创造的向社会主义过渡的经济政治条件,采取社会主义工业化和社会主义改造同时并举的方针,实行逐步改造生产资料私有制的具体政策,从理论和实践上解决了在中国这样一个占世界人口近四分之一的、经济文化落后的大国中建立社会主义制度的艰难任务。
%     \item 第二,毛泽东同志提出的对人民内部的民主方面和对反动派的专政方面互相结合起来就是人民民主专政的理论,丰富了马克思列宁主义关于无产阶级专政的学说。
%     \item 第三,在社会主义制度建立以后,毛泽东同志指出,在社会主义制度下,人民的根本利益是一致的,但人民内部还存在着各种矛盾,必须严格区分和正确处理敌我矛盾和人民内部矛盾。
%     \item 第四,他提出人民内部要在政治上实行‘团结———批评———团结’,在党与民主党派的关系上实行‘长期共存、互相监督’,在科学文化工作中实行‘百花齐放、百家争鸣’,在经济工作中实行对全国城乡各阶层统筹安排和兼顾国家、集体、个人三者利益等一系列正确方针。
%     \item 第五,他多次强调不要机械搬用外国的经验,而要从中国是一个大农业国这种情况出发,以农业为基础,正确处理重工业同农业、轻工业的关系,充分重视发展农业和轻工业,走出一条适合我国国情的中国工业化道路。他强调在社会主义建设中要处理好经济建设和国防建设,大型企业和中小型企业,汉族和少数民族,沿海和内地,中央和地方,自力更生和学习外国等各种关系,处理好积累和消费的关系,注意综合平衡。
%     \item 第六,他还强调工人是企业的主人,要实行干部参加劳动、工人参加管理、改革不合理的规章制度和技术人员、工人、干部‘三结合’。
%     \item 第七,他提出了调动一切积极因素,化消极因素为积极因素,以便团结全国各族人民建设社会主义强大国家的战略思想。
% \end{itemize}

\nocite{*}
\printbibliography[heading=bibliography,title=参考文献]
\end{document}
