% Created 2021-10-06 Wed 23:15
% Intended LaTeX compiler: xelatex
\documentclass[11pt]{article}
\usepackage{xeCJK}
\usepackage{graphicx}
\usepackage{longtable}
\usepackage{wrapfig}
\usepackage{rotating}
\usepackage[normalem]{ulem}
\usepackage{amsmath}
\usepackage{amssymb}
\usepackage{capt-of}
\usepackage[hidelinks]{hyperref}
\usepackage{biblatex}
\addbibresource{/Users/dcy/Desktop/inbox/projects/reference.bib}
\author{董晨阳}
\date{\textit{<2021-10-06 Wed>}}
\title{为何毛泽东思想在年轻人中再次流行}
\hypersetup{
 pdfauthor={董晨阳},
 pdftitle={为何毛泽东思想在年轻人中再次流行},
 pdfkeywords={},
 pdfsubject={},
 pdfcreator={Emacs 27.2 (Org mode 9.5)}, 
 pdflang={English}}
\begin{document}

\maketitle
\tableofcontents

\section{引言}
\label{sec:org86cab2f}
\subsection{建国以来毛泽东思想的流行趋势}
\label{sec:orga3300ff}


\paragraph{}

在 21 世纪以前,毛泽东思想的流行趋势可以从毛选的发行数量上略知一二。



\subsection{界定“毛泽东思想”}
\label{sec:org6ff9481}
\section{现实因素对年轻人的影响}
\label{sec:orga01109d}
\subsection{西方形象的瓦解}
\label{sec:orgf5991e4}
\subsection{现实经济社会环境}
\label{sec:org0d9b11d}

\section{毛泽东思想的生命力}
\label{sec:org1296cfe}

\section{毛泽东思想再次流行的意义}
\label{sec:orgd6bc97a}

\section{参考文献}
\label{sec:org61a5111}
\nocite{*}
\printbibliography
\end{document}
