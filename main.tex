\documentclass[a4paper,12pt]{ctexart}
% \usepackage{xeCJK}
\usepackage{amsmath}
\usepackage{amsfonts}
\usepackage{float}
\usepackage{enumerate}
\usepackage{graphicx}
\usepackage{tikz}
\usepackage[scale=0.8]{geometry}
\usepackage[hidelinks]{hyperref}
\title{经济学综合第一次作业}
\author{董晨阳 2201211201}
\date{\today}
\begin{document}
\maketitle
\section{两期模型}
根据家庭最大化两期效用的 F.O.C.
\begin{eqnarray*}
    u'(c_0)&=\beta u'(c_1)(1+r)\\
    -\frac{v'(l_t)}{u'(c_t)}&=w_t, t=0,1
\end{eqnarray*}
可知
\begin{eqnarray}
    \frac{1}{c_0}&= \displaystyle\frac{\beta}{c_1} (1+r) \label{eq1}\\
    c_t&=w_t, t=0,1\label{eq2}
\end{eqnarray}
公司最大化利润
\begin{equation}
    A_t \alpha l_t^{\alpha-1}=w_t, t=0,1\label{eq3}
\end{equation}

均衡时产品市场出清,资本市场在第二期储蓄为0,劳动力市场均衡
\begin{eqnarray}
    A_tl_t^{\alpha}&=c_t,&t=0,1\label{eq4}\\
    b_1&=0&\nonumber\\
    l_t^d&=l_t^s,&t=0,1\nonumber
\end{eqnarray}
(\ref{eq1})(\ref{eq2})(\ref{eq3})(\ref{eq4})解得
\begin{eqnarray*}
    c_t= &w_t&=A_t\alpha^\alpha\\
    &r &=\frac{1}{\beta}-1\\
    &l_t &=\alpha
\end{eqnarray*}
\section{带有人力资本的索洛模型}
\subsection*{a}
总资源约束,以及实体资本、人力资本的运动方程分别为
\begin{eqnarray*}
    Y_t=C_t+s_kY_t+s_hY_t\\
    \dot{K}_t=s_kY_t-\delta_kK_t\\
    \dot{H}_t=s_hY_t-\delta_hH_t
\end{eqnarray*}

\subsection*{b}
新的运动方程为
\begin{eqnarray*}
    \dot{k}_t=s_ky_t-(\delta_k+n+g)k_t\\
    \dot{h}_t=s_hy_t-(\delta_h+n+g)h_t
\end{eqnarray*}

\subsection*{c}
稳态需满足$\dot{k}_t=\dot{h}_t=0$
解得
\begin{eqnarray}
    k^*=&\displaystyle(\frac{s_h^\beta s_k^{1-\beta}}{\delta+n+g})^{\frac{1}{1-\alpha-\beta}}\\
    h^*=&\displaystyle\frac{s_h}{s_k}k^*\label{eq:fuck}
\end{eqnarray}

\subsection*{d}
易看出$k^*(s_k)$是递增函数,故$s_k$的提升会提升稳态水平下的$k^*$

类似地根据对称性,由于$k^*(s_h)$是递增函数,故$h^*(s_k)$也是递增函数,故$s_k$的提升也会提升稳态水平下的$k^*$

对于比值,由(\ref{eq:fuck})可看出,$k^*/h^*=s_k/s_h$也会提升

\subsection*{e}
平衡增长路径上$\dot{k}=\dot{h}=0$,因此
\begin{eqnarray*}
    \dot{K}=\dot{(kAL)}=\dot{k}+\dot{A}+\dot{L}=n+g\\
    \dot{H}=\dot{(hAL)}=\dot{h}+\dot{A}+\dot{L}=n+g
\end{eqnarray*}

\subsection*{f}
平衡增长路径上,根据生产函数的齐次性
\begin{eqnarray*}
    \dot{Y}=&\dot{F(K,L,A,H)}&=\alpha\dot{k}+\beta\dot{L}+(1-\alpha-\beta)(\dot{A}+\dot{L})=n+g\\
    \dot{Y/N}=&\dot{y}-\dot{N}&=g
\end{eqnarray*}

\subsection*{g}
\begin{minipage}[H]{0.48\linewidth}
    \begin{figure}[H]
        \centering
        \begin{tikzpicture}
            \draw[->] (0,0)--(5.2,0);
            \draw[->] (0,0)--(0,5.2);
            \draw[dashed] (0,3)--(5,3);
            \draw (0,3)--(1,3)--(1,1);
            \draw (1,1) arc (145:90:4.5);
        \end{tikzpicture}
        \caption{$k$的变化}
    \end{figure}
\end{minipage}
\begin{minipage}[H]{0.48\linewidth}
    \begin{figure}[H]
        \centering
        \begin{tikzpicture}
            \draw[->] (0,0)--(5.2,0);
            \draw[->] (0,0)--(0,5.2);
            \draw[dashed] (0,3)--(5,3);
            \draw (0,3)--(1,3);
            \draw (1,3) arc (225:305:2)--(5,3);
        \end{tikzpicture}
        \caption{$h$的变化}
    \end{figure}
\end{minipage}

\begin{minipage}[H]{0.48\linewidth}
    \begin{figure}[H]
        \centering
        \begin{tikzpicture}
            \draw[->] (0,0)--(5.2,0);
            \draw[->] (0,0)--(0,5.2);
            \draw[dashed] (0,3)--(5,3);
            \draw (0,3)--(1,3)--(1,1);
            \draw (1,1) arc (225:270:1)arc(165:90:3);
        \end{tikzpicture}
        \caption{$c$的变化}
    \end{figure}
\end{minipage}
\end{document}
