% Intended LaTeX compiler: xelatex
\documentclass{ctexart}
\usepackage{xeCJK}
\usepackage[hidelinks]{hyperref}
\usepackage[style=gb7714-2015ay]{biblatex}
\usepackage{indentfirst}
\usepackage{graphicx}
\usepackage{longtable}
\usepackage{wrapfig}
\usepackage{rotating}
\usepackage[normalem]{ulem}
\usepackage{amsmath}
\usepackage{amssymb}
\usepackage{capt-of}
\usepackage[export]{adjustbox}
\usepackage{url}
\usepackage{subfigure}
\addbibresource{reference.bib}
\title{为什么年轻人又开始读《毛选》了}
\author{董晨阳\footnote{经济学院,学号 1800015446}}
\date{\today}
\begin{document}
\maketitle
\tableofcontents
\begin{abstract}
	近年来,年轻人中掀起了读《毛泽东选集》的热潮。本文试图分析这股热潮形成的原因。本文认为,一方面年轻人面临现实因素迫切希望科学理论指导,另一方面毛泽东思想有内在的强大的生命力,这两方面因素共同促成了年轻人读《毛泽东选集》,并尝试对这股热潮的意义进行评析。

	{\bf\emph{ 关键词:毛泽东思想;年轻人;  }\rm}
\end{abstract}
\newpage
\section{引言}
\subsection{年轻人与《毛选》}
《毛泽东选集》这本初次出版于近 50 年前的书,
在年轻人群体中又有了新的流行趋势。
如图\ref{fig:subway}所示,地铁上都能看到不少人捧着《毛选》研读。
甚至于以二次元文化著称的 b 站上,
如图\ref{fig:bilibili}所示的朗读《毛选》视频重置后仍有近百万播放量,
并能保持持续的热度。毫无疑问,年轻人中产生了新的《毛选》热。
这让我思考,为什么年轻人又开始热衷读《毛选》了
\begin{figure}
	\subfigure[地铁上读《毛选》的人]{
		\includegraphics[width=0.45\textwidth]{lib/地铁上的读书人.jpg}
		\label{fig:subway}
	}
	\subfigure[bilibili 朗读《毛选》视频]{
		\includegraphics[width=0.45\textwidth]{lib/bilibili.png}
		\label{fig:bilibili}
	}
	\caption{年轻人中《毛选》热}
\end{figure}
\subsection{建国以来毛泽东思想的流行趋势}
在 21 世纪前,毛泽东思想的流行趋势可以用毛选的发行数来表征。

% \citeauthor{张慎趋2006毛泽东选集}在研究中发现,
自 1951 年 10 月《毛选》第一卷发行至到 1965 年末,
全国累计印制《毛选》1000 多万套。
而文革 10 年间对毛泽东思想的崇拜使得《毛选》非常热门,
《毛选》印刷量达 2.4 亿套
(包括少数民族文版、盲文版、外文版)。

文革之后一方面旧有印书量积压,
另一方面由于各种西方思潮影响,
到 20 世纪 80 年代,《毛选》在书店无人问津。
毛泽东思想仿佛被人遗忘在角落。
这种情况直至 1989 年才有所改善。
大学生们在历经纷繁复杂的思潮影响后,
开始寻找《毛选》阅读,呈现“寻找毛泽东热”,
《毛选》又一度脱销。

进入 90 年代,1991 年中央决定出版《毛选》第一卷至第四卷第二版,
一年内全国发行 1194 万套。

进入 21 世纪后,纸质传媒工具的影响力式微。
毛泽东思想的流行程度经历了一个先下降后上升的趋势。
加入世贸组织后更多的新思想、新现象涌入中国,
部分媒体和“公共知识分子”对我国社会现象进行了“反思”与“重构”,
社会主义的科学社会主义本质的奇谈怪论,一直不绝于耳,
在党内和社会上造成严重思想混乱。

党的十八大以来,
以习近平同志为核心的党中央以巨大的政治勇气和强烈的责任担当,
统揽伟大斗争、伟大工程、伟大事业、伟大梦想,
提出一系列新理念新思想新战略,出台一系列重大方针政策,
推出一系列重大举措,推进一系列重大工作,
解决了许多长期想解决而没有解决的难题,
办成了许多过去想办而没有办成的大事,
推动党和国家事业取得历史性成就、发生历史性变革,
调正了中国巨轮的航向。
毛泽东思想又逐渐重新流行起来。疫情以来流行程度更甚。
我爬取了年轻人聚集的网络问答平台知乎上
“就中国而言,近代到今最伟大的人是谁?”这一问题的
\footnote{原问题网址\url{https://www.zhihu.com/question/371931317}}
的 18000 余条回答,
部分统计结果如图\ref{zhihu}所示。
可见毛泽东思想在年轻人中又重新流行起来。

\begin{figure}
	\includegraphics[center,width=0.8\textwidth]{lib/zhihu.png}
	\caption{就中国而言,近代到今最伟大的人是谁?}
	\label{zhihu}
\end{figure}


\subsection{界定“毛泽东思想”}
这里需要特别说明的是,本文使用的“毛泽东思想”概念,是经中共中央第二个历史决议所重新界定了的概念,并不是利用所谓“毛泽东思想”为文革招魂。第二个历史决议定义的概念使得毛泽东思想同毛泽东同志的晚年错误从根本上划清了界限,从而为在改革开放新的历史条件下继续坚持和发展毛泽东思想、开创中国特色社会主义打开了指导思想上的通道。
% 即
% \begin{quote}
% 	\begin{itemize}
% 		\item 第一,毛泽东同志和中国共产党,依据新民主主义革命胜利所创造的向社会主义过渡的经济政治条件,采取社会主义工业化和社会主义改造同时并举的方针,实行逐步改造生产资料私有制的具体政策,从理论和实践上解决了在中国这样一个占世界人口近四分之一的、经济文化落后的大国中建立社会主义制度的艰难任务。
% 		\item 第二,毛泽东同志提出的对人民内部的民主方面和对反动派的专政方面互相结合起来就是人民民主专政的理论,丰富了马克思列宁主义关于无产阶级专政的学说。
% 		\item 第三,在社会主义制度建立以后,毛泽东同志指出,在社会主义制度下,人民的根本利益是一致的,但人民内部还存在着各种矛盾,必须严格区分和正确处理敌我矛盾和人民内部矛盾。
% 		\item 第四,他提出人民内部要在政治上实行‘团结———批评———团结’,在党与民主党派的关系上实行‘长期共存、互相监督’,在科学文化工作中实行‘百花齐放、百家争鸣’,在经济工作中实行对全国城乡各阶层统筹安排和兼顾国家、集体、个人三者利益等一系列正确方针。
% 		\item 第五,他多次强调不要机械搬用外国的经验,而要从中国是一个大农业国这种情况出发,以农业为基础,正确处理重工业同农业、轻工业的关系,充分重视发展农业和轻工业,走出一条适合我国国情的中国工业化道路。他强调在社会主义建设中要处理好经济建设和国防建设,大型企业和中小型企业,汉族和少数民族,沿海和内地,中央和地方,自力更生和学习外国等各种关系,处理好积累和消费的关系,注意综合平衡。
% 		\item 第六,他还强调工人是企业的主人,要实行干部参加劳动、工人参加管理、改革不合理的规章制度和技术人员、工人、干部‘三结合’。
% 		\item 第七,他提出了调动一切积极因素,化消极因素为积极因素,以便团结全国各族人民建设社会主义强大国家的战略思想。
% 	\end{itemize}
% \end{quote}
\section{现实因素对年轻人的影响}
\label{sec:2}
年轻人读《毛选》并非是闲暇时的消遣,
深层次原因是年轻人在现实中遇到了问题,希望得到科学理论的指导。
而产生问题的来源,一方面是旧有的源自西方的理论受到冲击,
另一方面则是新生的年轻人们需要面对的问题层出不穷。

\subsection{西方形象的瓦解}
改革开放以来,西方是我们追赶的对象。因此曾经会对西方有着一些美好的幻想。
过去西方的形象是民主的、自由的,倡导普世价值和全球化。
西方学者鼓吹的所谓美国梦正是这样的体现。
美国梦是民主、权利、自由、机会和平等的资产阶级幻想,
使人们相信任何人都有可能通过自己的努力迈向巅峰。

过去由于西方在经济领域的先进,
人们为了像西方一样富裕,在各个角度都向西方学习,因此这种思想非常有市场。
同时国内也有不少的“公知”为西方文化与意识形态站台,利用我党历史上犯过的错误,
大搞历史虚无主义,否定新中国前三十年的奋斗历程,描绘西方为人类文明的灯塔。
但自从中美贸易争端开始,逆全球化、民粹主义盛行,“灯塔”不亮了。
疫情更是撕下了西方各国间温情脉脉的面具,传统西方形象受到质疑。
旧有的西方价值观缺陷暴露出来,并被疫情的冲击放大。
人们识别了“美国梦”的幻影,
而所谓的“民主”、“自由”成就了香港动乱、黑人弗洛伊德等一长串闹剧。
西方形象瓦解。

\subsection{现实经济社会环境}

\begin{figure}
	\includegraphics[width=0.8\textwidth]{lib/GDP.jpg}
	\caption[Caption for LOF]{我国经济增长}
	\label{fig:GDP}
\end{figure}

除了西方形象的垮台,新的问题也算困扰年轻人的因素。
我国经济在步入新常态,经济增长从高速转为中高速后,
部分之前经济高速发展所掩盖的深层次问题逐渐暴露出来。

首当其冲的是贫富差距加大。
年轻人们努力奋斗,但却没有让自己的生活更美好,
反而新时代的资本家隐隐有重新出现的苗头。
因此年轻人自嘲为“打工人”,违反劳动法的 996 诙谐地被比作了“打工”。
他们在迷茫中希望找到思想指引,
让他们作为新时代的工人阶级,有一条没有剥削、可以实现自己劳动价值的道路。

其次是社会阶层出现固化的迹象。
“丧”和“躺平”等词的出现,显示出年轻人面对阶层固化的无奈。
日本就是前车之鉴。
在日本,政治家的儿子将成为政治家,银行家的儿子也会成为银行家。
这样下去,无论过多久,这个社会还是不会改变。
那里的年轻人只能无奈地作为资本家的蝼蚁被剥削被奴役。
我们的年轻人们自然不希望进入日本那样的低欲望社会,
因此希望能有指导他们面对这种现象抗争的理论。
老一辈革命家毁灭不公的旧世界,开创共产主义的新世界的理论使年轻人们着迷。

此外,新时代的资本家改头换面悄悄崛起。
他们有的勾结官员寻租枉法,有的买通喉舌宣扬“人民资本家”,
有的通过垄断打压其他企业。
自幼接受爱国主义教育的年轻人开始对这些现象产生怀疑,
希望得到合理的解释,因此他们会从社会主义的思想出发,
学习科学理论,冲破资本钩织的茧房。

\section{毛泽东思想的生命力}

\ref{sec:2}中的现实因素对年轻人的价值取向有很大的影响。
这些影响使得年轻人对毛泽东思想有天然的亲近。
而使毛泽东思想重新流行的最本质的因素,则是毛泽东思想有非常强大的生命力。

\subsection{完备科学的理论}

毛泽东思想是马克思主义中国化的重要源头,
是马克思主义中国化的第一个伟大理论成果。
而马克思主义包含着理论性与实践性、真理性与创新性、世界性与民族性的理论张力。
因而毛泽东思想是完备科学的理论,可以指导青年人的具体实践,有强大的生命力。

\subsubsection{理论性与实践性}

毛泽东思想和马克思主义一脉相承,其理论经得起历史唯物主义的验证,是科学的、
具有普适性的。而从实践上,毛泽东思想则充分考虑了中国的特殊条件,
将马克思理论与中国具体实际相结合,实现了马克思主义的第一次中国化。
从此社会主义在中国不是空洞的理论,而是完备的、理论与实践相统一的科学理论。
因此青年人可以通过毛泽东思想,指导行为实践,讲求理论与实践的统一。

\subsubsection{真理性与创新性}

毛泽东思想继承自马克思主义经典理论,
其真理性保证了毛泽东思想在任何时候都不过时。
而毛泽东思想也是马克思主义的进一步创新,实现了马克思主义中国化。
对矛盾、实践等有了更为深刻的解释。
因此青年人可以通过毛泽东思想,理解事物发展的客观规律,
理解历史唯物主义,旗帜鲜明地反对历史虚无主义,
作为行为的标尺。

\subsubsection{世界性与民族性}

毛泽东思想对中国具体实际有很深的理解,但也是属于世界的。
毛泽东思想启发了拉美、越南等地的共产主义运动,
更是深刻表明了毛泽东思想的先进性。
因此青年人可以通过毛泽东思想更深层次地理解中国现实,
把握我国的主要矛盾。

\subsection{实事求是、群众路线及独立自主的活的灵魂}

贯穿于毛泽东思想的实事求是、群众路线、独立自主的三个基本方面,
体现了毛泽东思想的根本立场、方法和观点。这三个方面是毛泽东思想的活的灵魂。

\subsubsection{实事求是}

“实事求是”是讲一切工作的思想方法、一切工作所应遵循的思想路线。
一切从实际出发,理论联系实际,实事求是,在实践中检验真理和发展真理。
要“向实际情况作调查”“没有调查,没有发言权”。
“实事求是”是毛泽东思想的精髓。“实事求是”在研究调查的过程中是不可缺少的。
因此毛泽东思想在不同时代背景下都是常为新的。
在如今信息爆炸的时代,“实事求是”是一种难能可贵的品质。
青年人在经历过诸多“反转”事件之后,对“实事求是”的理解更深一层。

\subsubsection{群众路线}

群众路线被毛泽东概括为“一切为了群众,一切依靠群众”“从群众中来,到群众中去”,
把党的正确主张变为群众的自觉行动。
人民群众是历史的创造者,毛泽东思想坚持走群众路线,
正是充分相信群众,紧紧依靠人民群众,
我党取得了长征、抗日战争、解放战争的胜利。
也正是因为充分相信群众,紧紧依靠人民群众,
我党在改革开放过程中取得了辉煌成就,保持着勃勃生机。

\subsubsection{独立自主}

独立自主,就是坚持独立思考,走自己的路。
曾经我国为了快速发展有一段“造不如买,买不如租”的时期。
西方贸易理论也支撑起这样做的合理性。
但中美贸易争端以来,国际形势愈发复杂,贸易理论黯然失色,
坚持独立自主也显得愈发重要,毛泽东思想的精髓也随之重新令人信服。

\section{《毛选》重又流行的意义}

鲁迅曾说,“愿中国青年都摆脱冷气,只是向上走,不必听自暴自弃者流的话。
能做事的做事,能发声的发声。有一分热,发一分光。”
毛主席也曾说“你们青年人朝气蓬勃,正在兴旺时期,好像早晨八九点钟的太阳。
希望寄托在你们身上。世界是属于你们的。中国的前途是属于你们的。”
年轻人与《毛选》相伴,正是青年人决心走向觉醒之路、破除历史虚无主义、
为共产主义建设发光发热的体现。

客观上的经济社会变化与主观的青年觉醒的思潮汇成时代的洪流。
而对于客观因素与主观意识之变的洞悉与深刻把握,
恰恰也正是毛选经久不衰的缘由所在。
比如《论持久战》、《论十大关系》、《矛盾论》、《实践论》、
《丢掉幻想、准备斗争》等名篇,
不仅切实指导了革命年代的中国共产党,
对于今天处在新的历史节点的中国而言,亦有重要意义。
青年群体中《毛选》再次流行,为我国社会主义建设的生力军提供了思想指引。
势必为我国实现社会主义现代化提供源源不断的动力。
\nocite{*}
\printbibliography[heading=bibliography,title=参考文献]
\end{document}
