% Created 2021-10-16 Sat 11:08
% Intended LaTeX compiler: xelatex
\documentclass[11pt]{ctexart}
\usepackage{xeCJK}
\usepackage{graphicx}
\usepackage{longtable}
\usepackage{wrapfig}
\usepackage{rotating}
\usepackage[normalem]{ulem}
\usepackage{amsmath}
\usepackage{amssymb}
\usepackage{capt-of}
\usepackage[hidelinks]{hyperref}
\author{董晨阳 \footnote{经济学院,1800015446}}
\date{\today}
\title{中国平安财报分析\\\large{保险资产管理第二次作业}}
\hypersetup{
 pdfauthor={董晨阳 \footnote{经济学院,1800015446}},
 pdftitle={中国平安财报分析 \large{保险资产管理第二次作业}},
 pdfkeywords={},
 pdfsubject={},
 pdfcreator={Emacs 27.2 (Org mode 9.5)}, 
 pdflang={English}}
\usepackage{biblatex}

\begin{document}

\maketitle
\tableofcontents

\newpage
\section{平安所持有金融资产的结构}
\label{sec:org68f553e}
截至2021年6月30日,中国平安保险资金投资组合规模近3.79万亿元,较年初增长1.2\%。中国平安近几年的资产负债表如图 \ref{fig:balancesheet} 所示\footnote{数据来源 wind}。

\begin{figure}[htbp]
\centering
\includegraphics[width=.9\linewidth]{./lib/平安资产负债表.png}
\caption{\label{fig:balancesheet}中国平安资产负债表}
\end{figure}

中国平安于2018年1月1日起执行新金融工具会计准则。其中,平安持有的金融资产结构如图 \ref{fig:portfolio}所示。占比最大的为以摊余成本计量的金融资产,达61.1\%。

\begin{figure}[htbp]
\centering
\includegraphics[width=.9\linewidth]{./lib/投资组合(按会计计量).png}
\caption{\label{fig:portfolio}投资组合(按会计计量)}
\end{figure}

具体到每一项投资品种,其账面值如图 \ref{fig:portfolio2} 所示。

\begin{figure}[htbp]
\centering
\includegraphics[width=.9\linewidth]{./lib/投资组合(按投资品种).png}
\caption{\label{fig:portfolio2}投资组合(按投资品种)}
\end{figure}

平安持有的金融资产主要集中在债券投资,占比达49.6\%。这其中主要是占比国债和地方政府债,体现了保险资金管理对安全性的要求。
\begin{figure}[htbp]
\centering
\includegraphics[width=.9\linewidth]{./lib/债券构成.png}
\caption{\label{fig:bond}债券构成}
\end{figure}

而占比2.4\%的公司债和信用债则91.0\%为AAA评级,从信用违约损失来看,平安保险资金投资的公司债券整体风险较小,稳健可控。

债权计划及债权型理财产品投资在总投资资产中占比12.1\%,外部信用评级95.0\%以上为AAA,,绝大部分项目都有担保或抵质押,很好地把握优质项目大量供给的黄金时期,有效提升整体组合的投资收益率。

股权方面,平安持有3005亿股票,861亿权益型资金,363亿信托公司信托计划、保险资产管理公司产品、商业银行理财产品,以及869亿非上市股权和1595亿长期股权投资。这部分资产主要是追求收益率与风险的平衡,体现资产管理的收益性要求

\section{平安投资风格}
\label{sec:org841b88f}
\subsection{长久期}
\label{sec:org7b3d3f1}
保险资金的负债是长久期的。2019 年,时任银保监人身险部副主任贾飙指出,我国人身险负债久期平均 12.44 年, 资产久期平均 5.77 年,缺口 6.67 年。为了对冲利率风险,需要配置长久期资产。

在长期资产供给紧缺的环境下,平安通过资产配置确保久期缺口维持在合理水平,持续优化保险资金资产负债匹配。平安积极配置大量国债、 地方政府债等长久期资产,使得保险资金组合可以长期维持 良好的资产负债久期匹配状态。

\begin{figure}[htbp]
\centering
\includegraphics[width=.9\linewidth]{./lib/每年新发债券久期构成.png}
\caption[Caption for LOF]{\label{fig:duration}每年新发债券久期构成\footnote{数据来源:wind}}
\end{figure}

如图\ref{fig:duration}所示,在目前长期限利率债市场容量约束下,平安积极布局优先股等广义久期资产,不断提升长久期资产配置,兼顾久期匹配和收益成本匹配的要求。
\begin{figure}[htbp]
\centering
\includegraphics[width=.9\linewidth]{./lib/平安债权计划及债权型理财产品结构和收益率分布.png}
\caption{平安债权计划及债权型理财产品结构和收益率分布}
\end{figure}

此外,平安还大量配置了房地产资产,如获取上海来福士广场等六个商业办公不动产项目60\%-70\%的股份等等。房地产收取租金的长期限进一步拉长了平安的资产久期。
\subsection{与公司业务互补性}
\label{sec:org3b3d83c}
平安是一个庞大的金融集团,涵盖了保险、银行、资管、科技四大业务。资管业务同时也考虑到对于其他业务特别是保险业务的协同性。

例如,平安通过重整方正集团,获取方正集团优秀的医疗体系,进而进一步服务于保险业务的人身险和养老险,进一步深化医疗健康产业战略布局、积极打造医疗健康生态圈。

又如,平安投资大量地产公司,为养老服务提供了充足的支撑。

\subsection{重视成本收益匹配风险管理}
\label{sec:orgb789e09}
平安高度重视成本收益匹配风险管理,设置以成本收益匹配为核心量化指标的风险偏好体系,并进行定期跟踪检视以及严格的压力测试,将其内置于大类资产配置流程,前置风险管理,并在市场波动加大时显著提升压力强度及测试频率,确保在发生罕见市场冲击时保险资金投资组合的安全。

平安高度重视成本收益匹配风险管理,设置以成本收益匹配为核心量化指标的风险偏好体系,并进行定期跟踪检视以及严格的压力测试,将其内置于大类资产配置流程,前置风险管理,并在市场波动加大时显著提升压力强度及测试频率,确保在发生罕见市场冲击时保险资金投资组合的安全。

此外,平安持续强化投后管理能力建设,对被投企业经营进行有深度、有细度、有力度的投后管理,促进与被投企业的文化融合;并且通过科技手段赋能投后关键事项管理,持续优化风险预警。
\section{平安投资收益规模与构成}
\label{sec:org37b072d}
平安投资收益规模与构成如图\ref{fig:revenue}所示。
\begin{figure}[htbp]
\centering
\includegraphics[width=.9\linewidth]{./lib/平安投资收益.png}
\caption{\label{fig:revenue}平安投资收益}
\end{figure}

2021年上半年,平安保险资金投资组合投资收益率受资本市场波动及减值计提增加等因素综合影响,有所承压。平安持续优化保险资金资产负债匹配,积极把握权益市场波动机会,灵活开展权益投资操作,并增加优质另类资产投资,获取投资收益。其年化净投资收益率3.8\%,年化总投资收益率3.5\%。
\section{平安金融工具伴生的风险、性质与暴露情况}
\label{sec:org687e1ae}
\subsection{利率风险}
\label{sec:org4a89a53}
平安不断提升长久期资产配置,兼顾久期匹配和收益成本匹配的要求。但是仍存在部分资产与负债失配风险。
\subsection{信用风险}
\label{sec:orgb0465c6}
尽管平安债权型投资主要通过内部评级政策及流程对现有投资进行信用评级,选择具有较高信用资质的交易对手,并设立严格的准入标准,但是仍有踩雷的可能。

例如平安投资华夏幸福。华夏幸福2021年初陷入债务危机,平安作为华夏幸福的第二大股东(后续被动成为第一大股东)承担了非常大的损失。
\begin{figure}[htbp]
\centering
\includegraphics[width=.9\linewidth]{./lib/华夏幸福.png}
\caption{\label{fig:huaxiaxingfu}平安计提有关华夏幸福的资产减值}
\end{figure}

\subsection{流动性风险}
\label{sec:orgf776237}
平安持有的大多数金融资产如国债、证金债等都具有较好的流动性。但不排除部分黑天鹅事件导致平安持有的债券流动性减弱,难以及时变现。如AAA国企永煤违约后,河南地区的债券无法正常流通。
\subsection{操作风险}
\label{sec:org1f9cf42}
操作风险是指由不完善或有问题的内部程序、员工和信息科技系统,以及外部事件所造成损失的风险。操作风险包括法律风险,但不包括策略风险和声誉风险。平安在管理其业务时会面临由多种不同因素而产生的操作风险。平安通过建立及不断完善风险管理体系、规范政策制度、使用管理工具及报告机制、加强宣导培训等方法管控操作风险。
\end{document}
