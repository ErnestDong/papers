\documentclass[a4paper,12pt]{ctexart}
\usepackage{amsmath}
\usepackage{amsfonts}
\usepackage{float}
\usepackage{enumerate}
\usepackage{svg}
\usepackage{graphicx}
\usepackage{booktabs}
\usepackage[hidelinks]{hyperref}
\usepackage[style=gb7714-2015ay]{biblatex}
\addbibresource{ref.bib}
% 【标题】
% 简明扼要,重点突出
% 【案例概要】
% 对案例进行整体论述(字数300-500字),说明案例所处领域,案例在交易、方案安排等方面的特点总结和创新之处。
% 【关键词】
% 3到5个
% 【案例正文】
% 1、结合案例情况,对案例进行分析,内容一般包括但不限于交易概述、交易各方、交易背景与目的、交易方案和架构、风险控制措施、交易影响、交易难点和特点分析(案例评价)等方面,希望同学根据自己的思考进行分析写作。
% 2、尽可能还原案例实操流程图,并分析实操过程中存在的主要问题、关键难点等。
% 3、注意融入产业、行业、企业等层面的分析,尽可能还原案例发生时的宏观经济环境、监管政策、行业环境和企业状况,并涵盖金融、法律、会计、税收等层面的创新与实践。
% 【附录】
%   如有数据图、表格等篇幅较大的非核心内容,但是又不可或缺的,不宜放在正文,宜放在附录。
% 【参考文献】
%  列出撰写案例的参考文献。
% 【选题参考范围】
% 案例的选题领域请参考本课程相关内容,请在其中选择一种类型进行案例分析,具体案例可选课内涉及的案例,也可选择同学本人关注的案例。请注意案例时效性,所选案例发生时间要求为三年以内。


\title{FBTC:比特币现货ETF上市分析}
\author{董晨阳}
\date{\today}

\begin{document}
\maketitle
\begin{abstract}

    2024年1月10日,美国SEC正式批准了11种比特币现货ETF,富达Fidelity的FBTC名列其中。Fidelity Wise Origin 比特币基金(FBTC)是在 Cboe BZX 交易所上市交易。该ETF的投资目标是跟踪现货比特币的价格,通过富达比特币参考利率指数的表现来衡量,该指数是使用来自符合条件的比特币现货市场的比特币价格和成交量加权中位价格(“VWMP”)方法构建的,该方法根据滚动 60 分钟增量的 VWMP 现货市场数据每 15 秒计算一次。

    此项交易的创新在于引入了在直接参与比特币价格波动收益的方式,而非利用衍生品来复制比特币价格波动。过去的比特币ETF如GBTC往往是采用信托+比特币的方式隔离风险,仅以私募形式在场外提供,面向机构和合格投资者开放。这也导致GBTC交易价格经常比基础比特币的实际价值大幅溢价或折价。而FBTC等ETF为所有投资者提供了通过传统经纪账户进入比特币市场的机会,而没有潜在的进入障碍,也不用直接持有比特币或开采比特币所涉及的风险。

\end{abstract}
\section{交易概述}
比特币\citep{nakamoto2008bitcoin}自2008年发明以来

2013年7月,Winklevoss兄弟提交了首个比特币ETF申请,随后多家公司纷纷效仿,加密货币基金灰度(Grayscale Investments)也申请将旗下Grayscale比特币信托(GBTC)转化为ETF,但美国证券交易委员会(SEC)均以“容易受到市场操纵”为由驳回了这些申请。

10年后,转机终于出现。2023年8月,美国华盛顿特区巡回上诉法院认为,SEC拒绝Grayscale Investments的比特币现货ETF申请是错误的,称SEC的决定是“武断和反复无常的”,未能解释其对比特币期货ETF和现货ETF的不同处理方式,迫使SEC重新考虑其立场。

事实上,去年的历史性判决已经暗示,比特币ETF的获批大概率只是时间问题。因此在法院对Grayscale申请案的判决后,比特币已上涨了约70%。

2024年伊始,屡败屡战的比特币ETF申请终于获批。美东时间1月10日,SEC主席詹斯勒(Gary Gensler)宣布,SEC批准了首批在美国上市的比特币ETF,包括贝莱德、富达、景顺、VanEck等11家公司申请的比特币ETF获得通过。
\section{交易各方}

\section{交易背景与目的}
\subsection{比特币简述}
\subsection{现货比特币ETF的努力}
\subsection{比特币现货ETF的意义}

\section{交易方案和架构}
\subsection{指数跟踪方式}
\subsection{资金托管方式}

\section{风险控制措施}

\section{交易影响}

\section{案例评价}

\nocite{*}
\printbibliography
\end{document}
