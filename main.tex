% 1、论文研究内容:可以尝试从以下方面入手(仅供参考,任选其一即可,或者另辟蹊径,但要贴合主题)
% (1)中国通过与相关国家共建“一带一路”,提高了国内各区域开放水平,拓展了对外开放领域,推动了制度型开放,构建了广泛的朋友圈,探索了促进共同发展的新路子,实现了同共建国家互利共赢。鼓励对“一带一路”国内相关省份或者沿线国家进行梳理对比,进行区域经济研究或国别分析。
% (2)从产业升级、绿色发展、安全发展、金融合作、贸易投资体系等与高质量共建“一带一路”的相关角度出发,展开深入研究。鼓励实证类和创新性研究,以小见大。
% (3)从高水平对外开放的角度出发,分析中国在国际上的比较优势和竞争优势,从马克思主义政治经济学角度研究高水平对外开放的理论逻辑与发展路径。
% 2、论文题目形式:题目可自拟,与主题高质量共建“一带一路”相关即可。
% 3、论文格式要求:论文需提交初稿和终稿,独立写作,不得抄袭。 本次需提交论文初稿
% 论文初稿要求确立论文的主要研究内容,提交完善的论文提纲以及文献综述。论文终稿需包括摘要、关键词、正文、参考文献等部分。其中论文正文字数要求至少6000字,应包括第一部分引言,主要阐述论文的研究背景与选题意义,及最后一部分结论,对论文的主要研究成果进行概述;中间部分根据研究内容自行安排。
% 4、论文引用要求:引用规范,请参考《经济研究》格式(http://www.erj.cn/cn/Info.aspx?m=20100913105301153616)
% 5、论文提交时间:论文初稿电子版请于11月24日(周四)上午10:00前提交至教学网—课程作业板块,提交时以添加附件的形式上传,不需要提交纸质版;论文终稿电子版请于12月29日(周四)上午10:00前提交至教学网—教学内容—作业板块,提交时以添加附件的形式上传,是否需要提交论文终稿纸质版再行通知,逾期不接受任何形式的补交。
% 6、论文文档命名方式:为便于分类收统,请同学们提交论文初稿及终稿时以“学号—学院(系、所)—专业—姓名—论文题目”为范式进行文档命名。
% 示例:“22XXXXXXX-经济学院-金融专业-张三-一带一路对中国产业升级的促进作用”
