\documentclass{ctexart}
\title{社会保险基金}
\begin{document}
\section{定义}
社会保险基金是国家依据法律政策规定,为保障劳动者暂时或永久丧失部分或全部劳动能力后的基本生活需要,在社会范围内向劳动者所在单位及个人征缴费用并集中起来的一种的专项基金。

社会保障基金包括社会保险基金、社会救济基金、社会福利基金、社会优抚基金等等。社会保险基金是社会保障基金的主要组成部分,也可以说是社会保障基金的主体,它主要包括养老保险基金、医疗保险基金、工伤保险基金、失业保险基金、生育保险基金等内容,它几乎能涵盖各方面群众的利益,影响十分广泛,地位十分重要。在国内,地方社保基金由单位、个人共同缴纳,政府进行财政补贴。我国的地方社保基金由当地的人社部门和财政部门共同管理,以财政部门为主。
\section{区别}
%http://www.gov.cn/xinwen/2016-03/28/content_5059217.htm
一是资金来源不同。全国社保基金由中央财政预算拨款、国有资本划转、基金投资收益和国务院批准的其他方式筹集的资金构成。社会保险基金是基本养老保险基金、基本医疗保险基金、工伤保险基金、失业保险基金和生育保险基金的统称,主要依靠所在单位和个人的缴费形成。

二是性质和用途不同。全国社保基金属于社会保障储备基金,专门用于人口老龄化高峰时期的养老保险等社会保障支出的补充、调剂,而社会保险基金则主要用于个人养老、医疗、工伤、失业和生育保险待遇的当期发放。

三是运营管理方式不同。全国社保基金由全国社保基金会管理运营,可投资配置于经国务院批准的固定收益类、股票类和未上市股权类等资产,可以在中国境内市场和境外市场投资运营。而各种社会保险基金则由各地社会保险经办机构管理,按社会保险险种分别建账,通过预算实现收支平衡。目前,社会保险基金必须存入财政专户,投资运营范围和领域受到严格限制。
\section{历史}
\subsection{改革开放前时期}

1978—1997年最主要的时代特征是由“企业保险和单位保障”转向“社会保险与养老保险费用社会统筹”,其过渡性历史色彩鲜明。1998—2009年最主要时代特征是在劳动与社会保障部全力推动下,初步形成“形式化”社会保险与社会保险基金体系框架。形式化主要是指有社会保险更多流于形式。2010—2019年最主要时代特征是人社部门主管社会保险基金体系与养老、医疗卫生服务体系形成平行的双轨结构,社会保险基金与医疗卫生服务行政主管部门之间部门博弈日趋激烈。

1949—1977年计划经济体制时期,中央政府主要根据东北解放区经验和苏联制度模式,迅速建立起社会主义的社会保险与社会保险基金体系,为社会主义制度建设奠定社会保险基础。1951年2月26日,中央人民政府政务院公布的《中华人民共和国劳动保险条例》,当时劳动保险基金会计制度和劳动保险基金行政管理工作由中华全国总工会负责,企业行政方面或资方按月缴纳的劳动保险金。

从1969—1997年,我们经历了基本无真正意义的社会保险基金阶段。直到改革开放后,中国社会保险体系改革才从失业保险起步,目的是配合国营企业破产和全面实行劳动合同制,推出了“国营企业职工待业保险”。职工待业保险基金由企业按照其职工标准工资总额1\%缴纳的额度,另加银行利息和地方财政补贴三部分组成。此后,我国社会保险体系逐渐由失业保险扩大到养老保险和医疗保险等。直到1994年劳动部发布《企业职工生育保险试行办法》,1996年劳动部颁布《企业职工工伤保险试行办法》后,到1990年代末期,失业、养老、医疗、生育、工伤社会保险体系框架才初步形成

20世纪90年代左右,我国养老保险制度开始引入个人账户,向统账结合的模式转变。1991年6月,国务院颁布《关于企业职工养老保险制度改革的决定》,首次明确提出“逐步建立基本养老保险与企业补充养老保险和个人储蓄性养老保险相结合制度”的目标,即基本养老保险基金按照“以支定收、略有结余、留有部分积累”的原则统一筹集。最重要的是,企业职工基本养老保险与养老保险基金制度设计成为其他社会保险险种设计模板,如医疗保险基金也采取统筹基金与个人账户相结合“统账模式”。总体来说,1978—1997年间,社会保险及社会保险基金体系正处于由传统的“单位保障”向“社会保险”体制转型阶段,是失业、养老、医疗、生育、工伤保险基金体系框架形成的萌芽期。1998年以来,由于城镇职工基本医疗保险制度设计存在若干结构缺陷与重大理论政策风险,例如:属地化筹资政策(即县级统筹层次)的“公正性”;社会医疗保险由谁管最好(社会医疗保险行政管理体制);个人账户和社会统筹基金的平衡;医疗待遇的准则问题等

1998—2009年的十多年间,有关社会保险基金政策法规体系框架逐步形成,其范围覆盖社会保险费征缴、支出、结余及其社会保险基金财务与会计制度等所有政策领域。1998年3月,劳动和社会保障部成立,拉开中国社会保险及社会保险基金建设时代的历史序幕,逐步建立起社会失业保险基金、社会养老保险基金、社会医疗保险基金、社会生育保险基金和社会工伤保险基金五大基金体系。这个时期政策法规的基本特征鲜明,反映出了社会保险基金财务会计管理制度发展的历史轨迹。比如社会保险费用筹资性质是“社会保险费”,而非“社会保障税”,1999年国务院以行政法规形式颁布的《社会保险费征缴暂行条例》。中国基本建立了类似社会保险形式的社会保险基金体系,社会保险费征缴、支付、结余、财务会计与核算体系、基金保值增值、基金监管、行政管理等社会保险全过程与所有构成要素体系框架基本形成,中国版社会保险财政学也应运而生

2010—2019年是社会保险与社会保险基金体系全面发展、结构转型与基本稳定期。随着2010年《中华人民共和国社会保险法》的出台,社会保险制度体系不断健全,新的社会保险险种陆续建立。但彼时社会保险体系中的城乡居民基本医疗保险、机关事业单位基本养老保险、城镇居民基本医疗保险、工伤保险、生育保险等险种尚缺乏相应会计制度对其核算进行明确规范,造成这些险种会计实务核算和加强日常财务管理中的实际困难。

自 2010 年后,中央政府有关社会保险基金的政策法规也发生若干重大变化。第一,全国社会保险基金政策法规决策和立法主体显著提高,中共中央、国务院颁布的文件数量激增。第二,全国社会保险基金政策法规涉及范围、内容、主题和重点日益多样。第三,全国社会保险基金预算编制、预算管理、预算审查监督和预算绩效管理等主题鲜明。第四,全国社会保险基金预算、预算管理与医药卫生体制改革、控制医疗费用等关系紧密。
%https://kns.cnki.net/KCMS/detail/detail.aspx?dbcode=CJFD&filename=BZLD199612016
\section{现状}
\end{document}
