\documentclass[a4paper,12pt]{ctexart}
% \usepackage{xeCJK}
\usepackage{csvsimple}
\usepackage{amsmath}
\usepackage{amsfonts}
\usepackage{float}
\usepackage{enumerate}
\usepackage{diagbox}
\usepackage{graphicx}
\usepackage[scale=0.8]{geometry}
\usepackage[hidelinks]{hyperref}
\title{保险精算第二次作业}
\author{董晨阳 2201211201}
\date{\today}
\begin{document}
\maketitle
\section{第一题}
以地区1行业1为基元,新费率水平于2020年1月1日开始实施,保单有效期为一年,新费率标准适用期限为一年。因此假定新费率实行期间损失发生的平均时间为2021年1月1日。

按照平均法计算发展因子有
\begin{eqnarray*}
    f_1&=(\frac{1210000}{1100000}+\frac{1430000}{1300000})/2&=1.1 \\
    f_2&=(\frac{1050000}{1000000}+\frac{1270500}{1210000})/2&=1.05
\end{eqnarray*}
累计已发生损失为
$$800000+184000+400000+230000 = 1614000$$
因此2017事故年的终极赔付额折算到2021年1月1日为
$$1614000 \times 1.05 \times 1.10\times 1.08^{3.5} = 2440443$$
按照损失成本法计算,指示费率的加权平均为
\begin{equation}
    \bar{IR}=\frac{2440443}{(10000+2000+4000+2000)\times 0.8} = 169.4752083\label{eq:ir}
\end{equation}

地区1的修正总风险量为$10000\times 1+4000\times 1.2=14800$,预计损失成本为
\begin{equation}\label{eq:1}
    (800000+400000)\times 1.05\times 1.1\times 1.08^{3.5} / 14800 = 122.598
\end{equation}

同理地区2的修正总风险量为$2000\times 1+2000\times 1.2=4400$,预计损失成本为
\begin{equation}\label{eq:2}
    (184000+230000)\times 1.05\times 1.1\times 1.08^{3.5} / 4400 = 142.270
\end{equation}

由(\ref{eq:1})(\ref{eq:2})可知,地区指示相对数分别为地区1为$1.0$,地区2为$142.270/122.598 = 1.160$

类似地,行业1的预计损失成本为
\begin{equation}\label{eq:3}
    (800000+184000)\times 1.05\times 1.1\times 1.08^{3.5}/(10000\times 1+2000\times 1.1) = 121.955
\end{equation}

行业2的预计损失成本为
\begin{equation}\label{eq:4}
    (400000+230000)\times 1.05\times 1.1\times 1.08^{3.5}/(4000\times 1+2000\times 1.1) = 153.643
\end{equation}

由(\ref{eq:3})(\ref{eq:4})可知,行业1的指示相对数为$1.0$,行业2的指示相对数为$153.643/121.955 = 1.260$

从而可以得到指示相对数的加权平均
\begin{equation}
    \label{eq:id}
    \bar{ID}=\frac{10000\times 1.0+4000\times 1.160+2000\times 1.260+2000\times 1.160\times 1.260}{10000+2000+4000+2000} = 1.115733
\end{equation}

地区1行业1是基元,由(\ref{eq:ir})与(\ref{eq:id})可知指示费率为
\begin{equation}
    IR_B=\frac{\bar{IR}}{\bar{ID}}=151.8958
\end{equation}

从而可知不同地区不同行业的指示费率如表\ref{tab:1}所示
\begin{table}[H]
    \centering
    \begin{tabular}{|c|c|c|}
        \hline
            & 地区1                               & 地区2                                           \\\hline
        行业1 & $151.8958$                        & $151.8958\times1.160 = 176.1991$              \\
        行业2 & $151.8958\times 1.260 = 191.3887$ & $151.8958\times1.160 \times 1.260 = 222.0109$ \\\hline
    \end{tabular}
    \caption{费率}\label{tab:1}
\end{table}
\section{第二题}
以地区1行业1为基元,记$i$地区$j$行业的已获风险量为$e_{ij}$。由不同地区各个行业已获风险量比例都相同的假设可知
\begin{eqnarray*}
    e_{11}&=500\times \frac{600}{1000}&=300\\
    e_{21}&=300\times \frac{600}{1000}&=180\\
    e_{31}&=200\times \frac{600}{1000}&=120\\
    e_{1j}&=500\times \frac{100}{1000}&=50\\
    e_{2j}&=300\times \frac{100}{1000}&=30\\
    e_{3j}&=200\times \frac{100}{1000}&=20\\
\end{eqnarray*}
其中$j\neq 1$。

每个地区各行业相对数的加权平均为
\begin{equation}
    \frac{1\times e_{i1}+1.25\times e_{i 2}+1.5\times e_{i 3}+1.75\times e_{i 4}+2\times e_{i 5}}{\sum_j e_{ij}}=1.25
\end{equation}

利用赔付率法计算地区间的指示相对数,结果如表\ref{tab:2}所示
\begin{table}[H]
    \centering
    \begin{tabular}{|c|c|c|c|}
        \hline
        地区 & 行业1的费率                   & 当前相对数                         & 指示相对数                                         \\\hline
        1  & $\frac{190}{1.25}=152$   & $1.0$                         & $1.0$                                         \\
        2  & $\frac{163}{1.25}=130.4$ & $\frac{130.4}{152}=0.8578947$ & $0.8578947\times\frac{0.62}{0.72}=0.7387426$  \\
        3  & $\frac{130}{1.25}=96$    & $\frac{96}{152}=0.6315789$    & $0.6315789\times \frac{0.77}{0.72}=0.6754385$ \\\hline
    \end{tabular}
    \caption{地区指示相对数}\label{tab:2}
\end{table}

从而得到各地区各行业的指示相对数加权平均为
\begin{equation}\label{eq:id_2}
    \bar{ID}=
    \begin{pmatrix}
        1 & 1.25 & 1.5 & 1.75 & 2
    \end{pmatrix}
    \begin{pmatrix}
        e_{11} & e_{21} & e_{31} \\
        \vdots & \vdots & \vdots \\
        e_{15} & e_{25} & e_{35} \\
    \end{pmatrix}
    \begin{pmatrix}
        1 \\0.7387426\\0.6754385
    \end{pmatrix}
    =1.07089
\end{equation}
各地区各行业的当前相对数加权平均为
\begin{equation}\label{eq:cd}
    \bar{CD}=
    \begin{pmatrix}
        1 & 1.25 & 1.5 & 1.75 & 2
    \end{pmatrix}
    \begin{pmatrix}
        e_{11} & e_{21} & e_{31} \\
        \vdots & \vdots & \vdots \\
        e_{15} & e_{25} & e_{35} \\
    \end{pmatrix}
    \begin{pmatrix}
        1 \\0.8578947\\0.6315789
    \end{pmatrix}
    =1.104605
\end{equation}

(\ref{eq:id_2})(\ref{eq:cd})可得基元的指示费率为
\begin{equation}
    IR_B=\frac{190}{1.25}\times 1.07\times \frac{\bar{CD}}{\bar{IR}}=167.76
\end{equation}
从而$i$地区$j$行业的费率$a_{ij}$为
\begin{eqnarray}
    \begin{pmatrix}
        a_{11} & \dots & a_{15} \\
        a_{21} & \dots & a_{25} \\
        a_{31} & \dots & a_{35} \\
    \end{pmatrix}&=
    167.76
    \begin{pmatrix}
        1 \\0.7387426\\0.6754385
    \end{pmatrix}
    \begin{pmatrix}
        1 & 1.25 & 1.5 & 1.75 & 2
    \end{pmatrix}\\
    & =
    \begin{pmatrix}
        167.76 & 209.70 & 251.64 & 293.58 & 335.52\\
        123.93 & 154.91 & 185.90 & 216.88 & 247.86\\
        113.31 & 141.64 & 169.97 & 198.30 & 226.62
    \end{pmatrix}
\end{eqnarray}
\end{document}
