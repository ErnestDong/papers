\documentclass[a4paper,12pt]{ctexart}
\usepackage{amsmath}
\usepackage{amsfonts}
\usepackage{float}
\usepackage{enumerate}
\usepackage{graphicx}
\usepackage{tikz}
\usepackage{csvsimple}
\usepackage[scale=0.8]{geometry}
\usepackage[hidelinks]{hyperref}
\title{保险精算第三次作业}
\author{董晨阳 2201211201}
\date{\today}
\begin{document}
\maketitle
\section*{第一题}
\subsection*{a}
累计赔付额如表\ref{tab:2}所示
\begin{table}[H]
    \centering
    \csvautotabular{data/累计赔付额1.csv}
    \caption{累计赔付额}\label{tab:2}
\end{table}
基于简单平均法计算发展因子如表\ref{tab:1}所示
\begin{table}[H]
    \centering
    \csvautotabular{data/f1.csv}
    \caption{发展因子}\label{tab:1}
\end{table}
利用链梯法可得每年的终极赔付额如表\ref{tab:3}所示
\begin{table}[H]
    \centering
    \csvautotabular{data/1a.csv}
    \caption{未决赔款准备金}\label{tab:3}
\end{table}
\subsection*{b}
赔付率法:预计终极赔付额为$1500 \times 0.6 = 900$,赔付率法的未决赔款准备金为$900-288 = 612$

B-F法:根据\ref{tab:1}及预计终极赔付额,计算未决赔款准备金为$900\times (\frac{1}{f_1f_2f_3})=645.23$
\section*{第二题}
\subsection*{a}
平均赔付额如表\ref{tab:4}所示
\begin{table}[H]
    \centering
    \csvautotabular{data/平均赔付额.csv}
    \caption{平均赔付额(元)}\label{tab:4}
\end{table}
同一年数据相加得$d_0^*=900,d_1^*=1509.091,d_2^*=1753.030$,因此有
\begin{eqnarray*}
    c \hat\lambda_2=&d_2^*&=1753.030\\
    \hat r_2=&P_{0,2}^*/n_0/c \hat\lambda_2&=0.114\\
    c \hat\lambda_1=&d_1^*/(1-\hat r_2)&=1703.432\\
    \hat r_1=&&=0.358\\
    c \hat\lambda_2=&&=1703.850\\
    \hat r_2=&&=0.528
\end{eqnarray*}

假设2021年后的日历年的年影响率为10\%,对$k>2$采用$$c\hat{\lambda}_k=c \hat\lambda_4 \times 1.1^{k-2}$$填充表\ref{tab:4}空下的位置有
\begin{eqnarray*}
    \hat P_{2020,2}=n_1\hat r_2c\hat\lambda_3=220000\\
    \hat P_{2021,1}=n_2\hat r_1c\hat\lambda_3=758733.3\\
    \hat P_{2021,2}=n_2\hat r_2c\hat\lambda_4=266200
\end{eqnarray*}
故终极赔付额如表\ref{tab:5}所示
\begin{table}[H]
    \centering
    \csvautotabular{data/终极赔付额.csv}
    \caption{终极赔付额}\label{tab:5}
\end{table}
\subsection*{b}
采用基本方法,准备金计算如表\ref{tab:6}所示
\begin{table}[H]
    \centering
    \csvautotabular{data/基本方法.csv}
    \caption{基本方法计算准备金}\label{tab:6}
\end{table}
采用比率方法,准备金计算如表\ref{tab:7}所示
\begin{table}[H]
    \centering
    \csvautotabular{data/比率方法.csv}
    \caption{比率方法计算准备金}\label{tab:7}
\end{table}
采用总赔付额方法,准备金计算如表\ref{tab:8}所示
\begin{table}[H]
    \centering
    \csvautotabular{data/总赔付额方法.csv}
    \caption{总赔付额方法计算准备金}\label{tab:8}
\end{table}

采用三种方法的假设不同。
\begin{itemize}
\item 基本方法的未决赔款准备金为未来各发展年的预计赔付额相加,
\item 比率方法则是进行了回溯,按照过去实际赔付额相较于理论赔付额的比例对理论进行调整。
\item 总赔付额方法则是假设总赔付额一定,总额与实际累计赔付额的差为未决赔款准备金。
\end{itemize}
\end{document}
