\documentclass[a4paper,12pt]{ctexart}
% \usepackage{xeCJK}
\usepackage{csvsimple}
\usepackage{indentfirst}
\usepackage{amsmath}
\usepackage{amsfonts}
\usepackage{float}
\usepackage{enumerate}
\usepackage{diagbox}
\usepackage{graphicx}
\usepackage{tikz}
\usetikzlibrary{calc}
\usepackage[scale=0.8]{geometry}
\usepackage[hidelinks]{hyperref}
\title{保险精算实务第一次作业}
\author{董晨阳 2201211201}
\begin{document}
\maketitle
\section*{第一题}
\subsection*{(a)}
赔付发生在日历年2015的事件只有事件2,因此已发生损失为$500+(1500-1000)=1000$
\subsection*{(b)}
赔付发生在日历年2016的事件有事件1、2、3,已发生损失如表\ref{(b)}所示,因此已发生损失总共为9000
\begin{table}[H]
    \centering
    \begin{tabular}{|r|l|}
        \hline
        事件1 & $1000+(0-0)=6000$    \\\hline
        事件2 & $1500+(0-1500)=0$    \\\hline
        事件3 & $2000+(1000-0)=3000$ \\\hline
    \end{tabular}
    \caption{2016日历年}
    \label{(b)}
\end{table}
\subsection*{(c)}
发生在2016年的事故有事故3和事故4,已发生损失如表\ref{(c)}所示,因此已发生损失总共为8000
\begin{table}[H]
    \centering
    \begin{tabular}{|r|l|}
        \hline
        事件3 & $2000+1000+0=3000$ \\\hline
        事件4 & $0+5000=5000$      \\\hline
    \end{tabular}
    \caption{2016事故年}
    \label{(c)}
\end{table}
\subsection*{(d)}
报告在2016年的事故有事故1、事故3,已发生损失如表\ref{(d)}所示,因此已发生损失总共为9000
\begin{table}[H]
    \centering
    \begin{tabular}{|r|l|}
        \hline
        事件1 & $1000+5000+0=6000$ \\\hline
        事件3 & $2000+1000+0=3000$ \\\hline
    \end{tabular}
    \caption{2016报告年}
    \label{(d)}
\end{table}
因此已发生损失总共为9000
\section*{第二题}
拟合数据趋势如表\ref{tab:2.1}所示,观察发现案均赔款有所增加,利用最小二乘法估计得
\begin{equation}
    Y=0.063823091X+9.309824398\label{eq:2.1}
\end{equation}
\begin{table}[H]
    \centering
    \csvautotabular{data/损失成本估计.csv}
    \caption{损失成本的估计值}
    \label{tab:2.1}
\end{table}

由(\ref{eq:2.1})可知,根据某年$X_0$对数损失成本为$Y_0$,可根据
\begin{equation}
    e^Y=e^{Y_0}e^{0.063823091(X-X_0)}\label{eq:2.2}
\end{equation}
估计损失成本,分别将2015年7月1日即$X=5.5$与后三年的数据代入(\ref{eq:2.2}),结果如表\ref{tab:2.2}
\begin{table}[H]
    \centering
    \csvautotabular{data/估计.csv}
    \caption{后三年的估计值与权重}
    \label{tab:2.2}
\end{table}
加权求和可得单位风险保单的损失成本约为15677.03182
\section*{第三题}
假定第4年底所有第0年事故年的所有赔案都赔付完毕,则根据表中数据利用平均法求发展因子如表\ref{tab:3.1}所示
\begin{table}[H]
    \centering
    \csvautotabular{data/发展因子.csv}
    \caption{发展因子}
    \label{tab:3.1}
\end{table}
则根据发展因子的定义可知
\begin{enumerate}
    \item 2015年的终极赔付额为第四年的累积赔付额,为$2023798$
    \item 2016年的终极赔付额为$2262863\times f_4=2332916.263$
    \item 2017年的终极赔付额为$2310585\times f_3f_4=2563572.894$
    \item 2018年的终极赔付额为$2384784\times f_2f_3f_4=3012647.503$
    \item 2019年的终极赔付额为$2327287\times f_1f_2f_3f_4=3423455.358$
\end{enumerate}
\end{document}
